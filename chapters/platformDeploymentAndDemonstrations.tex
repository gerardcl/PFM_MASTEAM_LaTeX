\chapter{Platform deployment and demonstrations}\label{H:platformDeploymentAndDemonstrations}

In order to demonstrate that the LiveMediaStreamer is a suitable tool to be used as the core framework of a cloud real-time media production platform it's required to showcase how it performs over the cloud. Therefore, it's important to demonstrate how it performs over the proposed architecture. So, previous chapters will implicitly help to deploy and demonstrate it.

\section{Platform deployment}

In order to demonstrate how LMS suits to become a proper tool it's proposed to deploy two scenarios with different complex degrees.

\begin{itemize}
\item Isolated deployment \hfill

The main goal of this deployment is to demonstrate how LMS performs inside a Docker container by comparing its performance in the same O.S. and machine but without running inside a container (i.e.: system installation).

In this scenario deployment LMS is configured to act as a transcoder service. This means applying one pipeline per stream type (i.e.: one video and one audio paths).

\item Generic scenario deployment \hfill

This scenario deployment aims to showcase a suitable and a as much generic as possible cloud real-time media production scenario. Therefore, it is proposed to configure LMS to receive eight streams (i.e.: four audio and four video streams), mix them and transmit them through RTP/RTSP and MPEG-DASH. 
\end{itemize}

\subsection{Isolated demonstrations}


\subsection{Generic scenario demonstration}

