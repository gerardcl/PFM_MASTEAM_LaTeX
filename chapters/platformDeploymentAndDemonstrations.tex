\chapter{Platform deployment and demonstrations}\label{H:platformDeploymentAndDemonstrations}

In order to demonstrate that the LiveMediaStreamer is a suitable tool to be used as the core framework of a cloud real-time media production platform it's required to showcase how it performs over the cloud. Therefore, it's important to demonstrate how it performs over the proposed architecture. So, previous chapters will implicitly help to deploy and demonstrate it.

\section{Platform deployment}

In order to demonstrate how LMS suits to become a proper tool it's proposed to deploy two scenarios with different complex degrees.

\begin{itemize}
\item Isolated deployment \hfill

The main goal of this deployment is to demonstrate how LMS performs inside a Docker container by comparing its performance in the same O.S. but without running inside a container (i.e.: system installation).

In this scenario deployment LMS is configured to act as a transcoder service. This means applying one pipeline per stream type (i.e.: one video and one audio paths).

\item Generic scenario deployment \hfill

This scenario deployment aims to showcase a suitable and a as much generic as possible cloud real-time media production scenario. Therefore, it is proposed to configure LMS to receive eight streams (i.e.: four audio and four video streams), mix them and transmit them through RTP/RTSP and MPEG-DASH. 
\end{itemize}

EXPLICAR L'ENTORN ON ES FARAN ELS DESPLEGAMENTS (MÀQUINA, CPU, RAM, OS,....)

\subsection{Isolated demonstrations}

In order to demonstrate results of interest what is done is to implement a C/C++ script which configures the LMS framework as shown in figure XXX. Moreover, in order to test how it performs the pipeline metrics are logged once per second (i.e.: pipeline losses and delay) and gathered by a Collectd client container properly configured. Then the Collectd client sends the data to the Graphite container. 

FIGURE XXX (MOSTRAR CODECS I PARÀMETRES, IDS, ETC...)

The Collectd client container, which reads from the volume where the LMS container is logging its metrics, is using the "tail" plugin (previously explained in chapter \ref{G:monitoringLayer}) with specific regular expressions, as shown next:

COLLECTD CLIENT CONFIGURATION




\subsection{Generic scenario demonstration}

