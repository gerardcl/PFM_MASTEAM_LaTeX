\cleardoublepage
\phantomsection
\chapter*{Introduction}

The audio-visual media content production industry (e.g.: broadcasters, small production companies, \ldots) has been, and already is, employing rigid and difficult to scale technologies to transport their streams through their processing chain. But, since early 2000s, a gradually adoption of IP technologies has been happening.

Although some control, management services and media content distribution over Internet (i.e.: streaming) or over proprietary operator networks (i.e.: IPTV) to stream live and VOD media to end-users, in parallel with the traditional broadcasting channels, almost all the post-production environments became file-based and can now be completely migrated to IP infrastructures. 

In that sense, only the live production environment has yet to be migrated to IP. Among several reasons, the inherent characteristics of high quality media streams present several nontrivial challenges for the current packet network technologies. For instance, there are some stringent constraints such as levels of synchronization, extremely low packet loss levels, high-bandwidth demand or jitter variation because these are not all intrinsically supported by current IP technologies.

Since few years ago, the broadcasting industry is pushing on to adopt IP as the transport technology because of several benefits such as:

\begin{itemize}
  \item Enhanced agility and flexibility of the broadcast workflows
  \item Convergence of services (i.e.: audio, video, metadata and general data)
  \item Format agnosticism. To support the adoption of coming new UHDTV formats such as 8K 
  \item Economy of scale by integrating Broadcasting industry into the far more massive IT industry
\end{itemize}

Recently, new network technologies have been appearing which could help to realise the media transport over packet networks. But, each one taken individually has some limitations which in practice make these solutions unsuitable to offer a complete end-to-end service. These new technologies and also industry initiatives want to address several of the previous challenges, but there is no complete set of guidelines and/or integral demonstration to validate the feasibility of high-quality production over IP yet.

Internet based services have been opening, since the early beginning of the Internet, new ways of cooperation, and now new industry sectors are migrating some of the processes, but also changing the production chain by adding new actors taking part on it. 

This project will provide a software based solution for the music sector, that supports the music producing industry (especially small and medium producers) to work with other professionals (e.g.: other music producers, engineers, creatives, musicians and clients) in a distributed and highly collaborative manner. This will be achieved by researching, developing and integrating three core, innovative cloud-based software modules in order to provide: a collaborative interface for music production (Digital Audio Workstation, or DAW), a suite of intelligent signal processing tools, and a complete videoconference system (VC) to facilitate communication and interaction between distributed nodes.

This is a first step to evolve from the traditional music production paradigm to a new, distributed one, thanks to the use of novel solutions from frontier research, using Internet and Cloud based technologies. The generic scenario is defined as a distributed virtual environment for music production connecting different nodes (e.g.: practise rooms in different venues, production studios, etc.) and enabling real-time low-latency interaction between them. The scalability of the proposed solution will bring a totally new approach to co-creation of music, in which all parts (content creators and clients) can interact at the same time. N-MusicPro will be initially validated in 3 real-scenarios (Radio production, advertising/branding and music rehearsals/electronic music live production).