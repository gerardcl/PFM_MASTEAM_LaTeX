\cleardoublepage
\phantomsection
\chapter*{Introducció}

Internet based services have been opening, since the early beginning of the Internet, new ways of cooperation, and now new industry sectors are migrating some of the processes, but also changing the production chain by adding new actors taking part on it. 

This project will provide a software based solution for the music sector, that supports the music producing industry (especially small and medium producers) to work with other professionals (e.g.: other music producers, engineers, creatives, musicians and clients) in a distributed and highly collaborative manner. This will be achieved by researching, developing and integrating three core, innovative cloud-based software modules in order to provide: a collaborative interface for music production (Digital Audio Workstation, or DAW), a suite of intelligent signal processing tools, and a complete videoconference system (VC) to facilitate communication and interaction between distributed nodes.

This is a first step to evolve from the traditional music production paradigm to a new, distributed one, thanks to the use of novel solutions from frontier research, using Internet and Cloud based technologies. The generic scenario is defined as a distributed virtual environment for music production connecting different nodes (e.g.: practise rooms in different venues, production studios, etc.) and enabling real-time low-latency interaction between them. The scalability of the proposed solution will bring a totally new approach to co-creation of music, in which all parts (content creators and clients) can interact at the same time. N-MusicPro will be initially validated in 3 real-scenarios (Radio production, advertising/branding and music rehearsals/electronic music live production).

