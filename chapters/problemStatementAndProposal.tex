\chapter{Problem statement and proposal}\label{B:problemStatementAndProposal}

As this title's chapter sample, the aim of this section is to provide a structured vision of the specific problems of each issue to be developed 




\section{Architecture study}
\subsection{Virtualization}

Virtualization Project Steps
After you’ve evaluated virtualization and want to move forward with it, it’s time to implement a virtualization plan. Don’t jump right in, the first steps are to create a virtualization project using these five steps:

Evaluate your current server workloads.

Determine whether virtualization can help you and figure out what your potential virtualization use cases might be.

Define your system architecture.

What form of virtualization will you use, and what kind of use case do you need to support?

Select your virtualization software and hosting hardware.

Carefully evaluate the virtualization software’s capabilities to ensure that it supports your use cases. Be sure to look at the new virtualization-enabled hardware systems.

Migrate your existing servers to the new virtualization environment.

Decide whether some of the new migration products can help you move your systems or if you need to move them manually — in either case, create a project plan to ensure everything is covered

Administer your virtualized environment.

Decide whether the virtualization product management tools are sufficient for your needs or whether you should look to more general system management tools to monitor your environment.

\subsection{Monitoring layer}
\section{Architecture proposal}
\section{Task planning}



