\chapter{State of the art}\label{A:stateOfTheArt}

This chapter is aimed to introduce current and tendencies status regarding real-time media production platforms. This is doing a differentiation regarding the economic possibilities, that gives us two different realities. The industry of the big broadcasters with high inversion possibilities and the small media production companies with low economic power, as might be regional broadcasters.

This will continue with how the IP convergence is giving the chance, not only to let the small media production companies to increase their production possibilities at a lower cost, but also to the big companies to add new values on data management and infrastructure scalability, reuse and reliability. In that sense, the concept of cloud migration will be focused in order to explain how IP convergence and related technologies open new possibilities.

This chapter is also intended to show how the items related to this project has ....

Finally, LiveMediaStreamer (LMS) framework, developed at the Audiovisual Unit of the i2CAT Foundation, is introduced in order to focus in which tool the project is mainly related to at application level.

\section{Real-time media production platforms}

Write down the main related differences

\subsection{The industry}

Highlight professional area and its tendences



Beginning with some control and management services and continuing with the media content
distribution over Internet (streaming) or over proprietary operator networks (IPTV) to stream live and
VOD media to end-users in parallel with the traditional broadcasting channels. Likewise, almost all
the post-production environments became file-based and are now completely migrated to IP
infrastructures. More recently, the contributions from high-quality remote feeds have been changing
the traditional baseband WAN links to IP-based ones.
Therefore, as broadcasting stands today, only the live production environment has yet to be migrated
to IP. Among several reasons, the inherent characteristics of broadcast-quality media streams present

\subsection{Small productions}

Highlight small production companies such regional broadcasters and "no professionals


\section{IP convergence}

Since early 00’s, a gradually adoption of IP technologies has been happening.

\subsection{Standards}


\subsection{Migration to the cloud}


\subsubsection{Hybrid cloud computing}


\subsubsection{Virtualization}


\subsubsection{Media streaming}


\subsubsection{Monitoring}


\section{Hype cycle}


\section{LiveMediaStreamer (LMS) framework}







