\chapter{State of the art}\label{A:stateOfTheArt}

Explain briefly the chapter's content 

\section{Real-time media platforms}

Write down the main related differences

\subsection{The industry}

Highlight professional area and its tendences



Beginning with some control and management services and continuing with the media content
distribution over Internet (streaming) or over proprietary operator networks (IPTV) to stream live and
VOD media to end-users in parallel with the traditional broadcasting channels. Likewise, almost all
the post-production environments became file-based and are now completely migrated to IP
infrastructures. More recently, the contributions from high-quality remote feeds have been changing
the traditional baseband WAN links to IP-based ones.
Therefore, as broadcasting stands today, only the live production environment has yet to be migrated
to IP. Among several reasons, the inherent characteristics of broadcast-quality media streams present

\subsection{Small productions}

Highlight small production companies such regional broadcasters and "no professionals


\section{IP convergence}

Since early 00’s, a gradually adoption of IP technologies has been happening.

\subsection{Standards}


\section{Migration to the cloud}


\subsection{Hybrid cloud computing}


\subsection{Virtualization}


\subsection{Media streaming}


\subsection{Monitoring}


\section{Hype cycle}


\section{LiveMediaStreamer (LMS) framework}







