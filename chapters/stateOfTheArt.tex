\chapter{State of the art}\label{A:stateOfTheArt}

This chapter aims to provide a vision of how the audio-visual media production has been, and still is, carrying out IP convergence. 

Moreover, once introduced the state of the art of the audio-visual media production issues, this chapter will be focused on the topics related on the production itself and specifically on the issues that this project is going to study.

...???...

\section{Audio-visual media production}

The audio-visual media content production industry, and more concretely the broadcasting one, has
been, and already is, employing rigid and difficult to scale technologies to transport their broadcast-
quality streams through their processing chain. But, since early 2000s, a gradually adoption of IP
technologies has been happening.

Although some control,management services and media content distribution over Internet (streaming) or over proprietary operator networks (IPTV) to stream live and VOD media to end-users, in parallel with the traditional broadcasting channels, almost all the post-production environments became file-based and can now be completely migrated to IP infrastructures. 

Therefore, as broadcasting stands today, only the live production environment has yet to be migrated
to IP. Among several reasons, the inherent characteristics of broadcast-quality media streams present several nontrivial challenges for the current packet network technologies. For instance, there are some
stringent constraints such as levels of synchronization, extremely low packet loss levels, high-
bandwidth demand or jitter variation because these are not all intrinsically supported by current IP
technologies (even high-end ones).

Since few years ago, the broadcasting industry is pushing on to adopt IP as the transport technology because of several forecasted benefits such as:

\begin{itemize}
  \item Enhanced agility and flexibility of the broadcast workflows
  \item Convergence of services (audio, video, metadata and general data)
  \item Format agnosticism. To support the adoption of coming new UHDTV formats such as 8K 
  \item Economy of scale by integrating Broadcasting industry into the far more massive IT industry)
\end{itemize}

Recently, new network technologies have been appearing which could help to realise the media
transport over packet networks, but each one taken individually, has some limitations which in
practice make these solutions unsuitable to offer a complete end-to-end service. These new
technologies and also industry initiatives want to address several of the previous
challenges, but there is no complete set of guidelines and/or integral demonstration to validate the
feasibility of high-quality production over IP yet.

\subsection{Technologies and standards}

\subsubsection{Hardware layer}

Ethernet (standardised in 1983) has been increasing its speed rate from the initial 10 Mbps to
100 Gbps (foreseen 400 Gbps by IEEE P802.3bs Task Force), with currently easily affordable 10 Gbps and 40 Gbps interfaces. These rates seem to be enough to accommodate current broadcast formats (HD 1,5 Gbps,
3G 3 Gbps and UHD 12 Gbps) and further innovations because the nature of the packet technologies
make them completely agnostic to the upper formats and indeed transparent for future formats in
contrast with current media transport technologies which are completely bounded with the transported
formats. But on another hand, Ethernet has not any timing awareness or QoS assurance, so it makes
difficult to accommodate current operation workflows over this technology. Nevertheless, because
Ethernet is widely used in the IT industry, its use as COTS switches have motivated studies about the
use of these switches in the broadcast industry deployments [EDW15] to validate specific necessary
features as the latency deviation or packet loss.

At the same level, to address some of the inherent Ethernet limitations, [AVB] appeared in 2011
which is a set of standard extensions to the Ethernet IEEE 802.1 focusing on timing and QoS
guarantee within local area networks. Its approach is a plug-and-play platform to ease transition from
current transport technologies to the newer ones using the same workflows, but the current version is
still limited to local premises and limited topologies. Since November 2012, because more varied
industry sectors joined the task group, a more general name, Time Sensitive Networks TSN, was
created to carry on with the new developments.

On another hand, the emergence of the SDN paradigm, separating the control and forwarding plane
besides creating northbound interfaces to interact with external applications enables new flexible and
customised network operation and deployments. There are a lot of foreseen benefits from this
approach but to be fully capable of support broadcast-quality streams some extensions should appear,
such as a specific extension which has been released by the ONF to address timing restrictions known
as OpenFlow Time Extension to OpenFlow 1.3.x ext340 [MIZ13].

Furthermore, the Telecom Industry introduced the [NFV] concept in 2012 to enable the shifting from
the hardware-centric approach to the software-centric one. In spite of not being thought for
broadcasting issues, this parading provides an amazing flexibility to bring up new deployments and
enhance current workflows and create future ones.

Related to above statements, the recent advances in chip designs, by industry leaders, such as Intel,
Broadcom, Xilinx, Altera, and others, have facilitated a strong movement towards consolidation of complex functions (such as encoding, transcoding, conversion, \ldots ) into a single device, instead
several disparate platforms. Additionally, these Hardware advances have expedited the use of
Software-centric frameworks, which provide for greater flexibility and customization from a business
standpoint. Coupled together, these advances will enable a broader adoption of upcoming media
technologies, at a reduced cost, without compromise on quality, flexibility and or capability. The
disruptive nature of such advances can be readily seen in the Mobile Telephony market, via the entry
of Open Source mobile phone software, such as Google’s Android OS, which when couple with low
cost but powerful chipsets, has democratised the market, allowing millions of low income countries
access to mobile phones and with it the Internet.

In parallel, regarding the specific timing and synchronising requirements of live media production,
the standard IEEE1588-2008 (revised in 2008), commonly known as PTPv2, appeared covering
several profiles to be used through several network environments. For instance, SMPTE published a
draft profile SMPTE ST 2059 (1 and 2) defining a reference alignment to SMPTE Epoch and there
are studies analysing the application of PTP to broadcast environments [SET15] under different
circumstances.

\subsubsection{Network layer}

In the network layer, IP is the de facto standard and within their protocol suite there are some solutions
which help to transport media content efficiently. For instance, IP supports multicast paradigm
operation using widely supported routing protocols (DVMRP, PIM, IGMP, etc.) but the
computational and scalable complexity of these protocols let to complex and limited deployments.
Regarding QoS, IP has a mechanism known as ToS/DSCP which marks the packets (in the header)
along their way to help mappings with lower-layer protocols (as Ethernet or MPLS) to implement
QoS at the buffer level. Furthermore, IP networks have been evolving its own architectures from
traditional hierarchic ones to flatter ones, such as the leaf-and-spine, used in most of the big data
centre deployments nowadays facilitating horizontal data movement, useful for heavy load
transactions between same level hosts.

On the transport layer, UDP has been preferred over TCP for real-time transport because of his
connectionless avoiding unnecessary retransmission for live streams. A basic extension, RTP (most
deployed version is RFC3550) was introduced to transport audio and media services using a
timestamp field together with the protocol for control purposes (RTCP). Recently, new extensions
have appeared introducing new header options to support the adoption of services related to media
production workflows. Concretely, [RTP1, RTP2, RTCP1] have been proposed to accommodate
media specific info over IP, answering to specific challenges.

\subsubsection{Upper layers}

Encapsulation audio (AES67-2013) and video (SMPTE 2022-6) standards have
appeared (2012-2013) to transport high-quality media signals over IP Networks, but specially in the
audio field a broad spectrum of proprietary solutions exist such as RAVENNA, Livewire or Dante.
On the video side, SMPTE 2022-6 is focused on mapping SDI and HDSDI (opposite to raw video,
audio and metadata mapping, known as essence mapping) within IP packets and further specific
solutions for manage packet loss recovery using FEC (SMPTE 2022-5) and a seamless protection
system (SMPTE 2022-7).

In the middle of 2014, the Video Services Forum (VSF) has formed a new group (SVIP) looking at
new encapsulation mechanisms for Audio, Video and Ancillary data into IP without using SDI
framing (raw data) to develop or recommend a standard for Video over IP without SDI encapsulation.
They aim to study and document the requirements for Video over IP/Ethernet within Plant (Video,
Audio, Ancillary Data, Bundles, Timing, Sequencing, Identities, and Latency), to research current
and proposed solutions to report on gaps between requirements and existing solutions (especially
regarding existing SMPTE 2022 Standards) and finally to propose scope for follow-on activity if
required

On another hand, the wide adoption of low-delay encoding (JPEG2K, AVC, AVCi, VC-2) for high-
quality video stream could represent a new opportunity to reduce the bandwidth consumption in
several scenarios. Likewise, high-compression mechanism as MPEG4 H264 or HEVC could be
useful to transport media content through very limited network resources scenarios (as Internet or
cloud-based systems).

Likewise, specific efforts have risen to arrange specific challenges such as a networked media
interface by Sony to carry a virtually lossless UHD/4K (12 Gbps non-compressed) over a single 10
GBE interface [SON14], limited proofs of concept [KOU13, POU14] or specific implementations
facing the switching-point issue [EDW14b]. All of these are a useful starting point for future
enhancements towards a global operational framework.

Another important issue is the automation of the system to enhance flexibility for deployment set-up
and maintenance. Here, some solutions as Zero-configuration networking [ZEROC] could contribute
using well-known protocols as DHCP or DNS-SD to enable auto-configuration and streaming
announcement, but to be implemented in an operational scenario a common approach should be
defined.

In the media plane, protocols as RTSP for end-to-end session control or SDP for service description
provide capabilities for the stream management. Complementary, media wrappers aim to gather
different types of programme media and associated information, as well as generically identify this
information. Different media wrapper formats are in use at this time, but, for the media industry, it is
important that the wrappers have characteristics like openness, extensibility, performance \ldots . The MXF (a SMPTE standard) is a “container” format which supports a number of different streams of coded
"essence", encoded in any of a variety of video and audio compression formats, together with a
metadata wrapper which describes the material contained within the MXF file enable interoperability
between different platforms. Also DDS (a machine-to-machines middleware standard from OMG)
could be used to enable interoperable media exchange between actors. At the same time, EBU has
launched the [FIMS] (Framework for Interoperable Media Services) which intends to answer to
different interoperability issues between SOA proprietary systems by defining an open, consensual
framework with standardised interfaces.

Regarding the measurement of media transport over IP, VSF published in 2006 a document [RVOIM]
to define the recommended metrics for Video over IP transport. The aim of the document is help in
the monitoring, troubleshooting, equipment performance compliance to standards and specifications,
verifying and measuring the delivered services statistics and equipment analysis and debug, but
basically with a distribution approach, no focused on live production as the current project aims.
As could be inferred from the above statements, there are no current common approach to solve the
whole challenge yet and to face this issue some initiatives have appeared lately. One of the first
outstanding initiatives was a BBC research project called IP Studio [BBC12] which tried to provide
an operational framework for a live studio within their environment, creating commercial outcomes
as Stagebox [STBOX] and being used in operational scenarios [RAW15].

Likewise, in 2013 SMPTE, VSF and EBU created the JT-NM task force [JTNM] to drive the
broadcasting industry towards a full IP adoption by providing guidelines to enable a successful
migration. Currently, the JT-NM is working to develop a reference architecture (Phase 2) to help all
the parties to speak a common language whereas defining specific requirements over concrete use
cases to uncover missing definitions to address the general scenario. Just in few months, the JT-NM
will publish the Phase 2 Final report by June 30, 2015 which will be taken into account to reuse as
many guidelines as possible in the project.

Recently, specific manufacturer approaches have been published [KOJ15] (and more are going to
appear in the near future) to address the requirements from the broadcasting industry, defining some
specific considerations and selecting standards (and drafts) to provide practical approaches.


\section{Migration to cloud}

\subsection{Hybrid cloud computing}


\subsection{Virtualization}


\subsection{Media streaming}


\subsection{Monitoring}


\section{LiveMediaStreamer (LMS) framework}







