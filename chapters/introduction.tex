\cleardoublepage
\phantomsection
\chapter*{Introduction}

The audio-visual media content production industry (e.g.: broadcasters, small production companies, \ldots) has been, and already is, employing rigid and difficult to scale technologies to transport and manage their streams through their processing chain. But, since early 2000s, a gradually adoption of IP technologies has been happening.

Despite some control, management services and media content distribution over Internet (i.e.: streaming), or over proprietary operator networks (i.e.: IPTV), to stream live and VOD media to end-users, in parallel with the traditional broadcasting channels, almost all the post-production environments became file-based and can now be completely migrated to IP infrastructures. 

Hence only the live production environment has yet to be migrated to IP. Among several reasons, the inherent characteristics of high quality media streams present several nontrivial challenges for the current packet network technologies. For instance, there are some stringent constraints such as levels of synchronization, extremely low packet loss levels, high-bandwidth demand or jitter variation because these are not all intrinsically supported by current IP technologies.

Since few years ago, the broadcasting industry is pushing on to adopt IP as the transport technology because of several benefits such as:

\begin{itemize}
  \item Enhanced agility and flexibility of the broadcast workflows
  \item Convergence of services (i.e.: audio, video, metadata and general data)
  \item Format agnosticism to support the adoption of coming new UHDTV formats such as 8K 
  \item Economy of scale by integrating broadcasting industry into the far more massive IT industry
\end{itemize}

But, there is still a lot of work to be done to achieve complete IP convergence, and lots of new possibilities thanks to different architectures and configurations that are offered by OTT (Over-The-Top) content management systems. 

Although the huge amount of issues to be solved, this project will only be focused on those related to cloud migration for real-time media production at application plane. 

To highlight that this project is part of the work done at my job position at the Audiovisual Unit of the i2CAT Foundation. Concretely, this project is one of the next steps related to the LiveMediaStreamer framework, an open-source software developed by my team. One of this main functionalities is the capability of working as a software-based audio and video mixer.


And these are:

 \begin{itemize}
  \item Virtualization layer \hfill \\
  \item Media layer \hfill \\
  \item Monitoring layer \hfill \\
\end{itemize}






