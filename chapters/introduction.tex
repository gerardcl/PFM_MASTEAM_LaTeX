\cleardoublepage
\phantomsection
\chapter*{Introduction}

The audio-visual media content production industry (e.g.: broadcasters, small production companies, \ldots) has been, and already is, employing rigid and difficult to scale technologies to transport and manage their streams through their processing chain. But, since early 2000s, a gradually adoption of IP technologies has been happening.

Despite some control, management services and media content distribution over Internet (i.e.: streaming) or over proprietary operator networks (i.e.: IPTV) to stream live and VOD media to end-users, in parallel with the traditional broadcasting channels, almost all the post-production environments became file-based and can now be completely migrated to IP infrastructures. 

In that sense, only the live production environment has yet to be migrated to IP. Among several reasons, the inherent characteristics of high quality media streams present several nontrivial challenges for the current packet network technologies. For instance, there are some stringent constraints such as levels of synchronization, extremely low packet loss levels, high-bandwidth demand or jitter variation because these are not all intrinsically supported by current IP technologies.

Since few years ago, the broadcasting industry is pushing on to adopt IP as the transport technology because of several benefits such as:

\begin{itemize}
  \item Enhanced agility and flexibility of the broadcast workflows
  \item Convergence of services (i.e.: audio, video, metadata and general data)
  \item Format agnosticism to support the adoption of coming new UHDTV formats such as 8K 
  \item Economy of scale by integrating broadcasting industry into the far more massive IT industry
\end{itemize}

Besides all steps done to achieve complete IP convergence there is still a lot of work to be done and lots of new possibilities thanks to different architectures and configurations that are offered by OTT (Over-The-Top) content management systems. 

This project will focus on application plane in order to analyse and propose solutions for issues related to cloud migration of media management. These are:

 \begin{itemize}
  \item Virtualization layer \hfill \\
  \item Media layer \hfill \\
  \item Monitoring layer \hfill \\
\end{itemize}






