\cleardoublepage
\phantomsection
\chapter*{Introduction}

The audio-visual media content production industry (e.g.: broadcasters, small production companies, \ldots) has been, and already is, employing rigid and difficult to scale technologies to transport and manage their streams through their processing chain. But, since early 2000s, a gradually adoption of IP technologies has been happening.

Despite some control, management services and media content distribution over Internet (i.e.: streaming), or over proprietary operator networks (i.e.: IPTV), to stream live and VOD media to end-users, in parallel with the traditional broadcasting channels, almost all the post-production environments became file-based and can now be completely migrated to IP infrastructures. 

Among several reasons, the live production environment has yet to be migrated to IP. And mainly due to the inherent characteristics of media streams that present several nontrivial challenges for the current packet network technologies. For instance, there are some stringent constraints such as levels of synchronization, extremely low packet loss levels, high-bandwidth demand or jitter variation because these are not all intrinsically supported by current IP technologies.

Since few years ago, the broadcasting industry is pushing on to adopt IP as the transport technology because of several benefits such as:

\begin{itemize}
  \item Enhanced agility and flexibility of the broadcast workflows
  \item Convergence of services (i.e.: audio, video, metadata and generic data)
  \item Format agnosticism to support the adoption of coming new UHDTV formats such as 8K 
  \item Economy of scale by integrating broadcasting industry into the far more massive IT industry
\end{itemize}

But, there is still a lot of work to be done to achieve complete IP convergence, and lots of new possibilities thanks to different architectures and configurations that can be applied on OTT (Over-The-Top) content management systems. 

Although the huge amount of issues to be solved, this project is focused on few issues related to cloud migration for real-time media production at application plane. Specifically, these are:

\begin{itemize}
\item Virtualization layer \hfill 

The virtualization paradigm offers specific solutions to improve scalability, robustness and reliability of any platform and services to be deployed over a cloud environment (n-tier applications development --REFERENCE--). 

\item Monitoring layer \hfill 

The monitoring layer is a crucial tool to be deployed which offers the capability of non-stop getting feedback from deployed services in order to real-time actuate over any defined alarm. This layer give also the chance to find and solve infrastructure/platform bottle-necks.

\item Application layer \hfill 

The application layer is the core service itself (the audio and video production core service) that must be adapted in order to offer full compatibility with previous mentioned layers and to become a cloud service.

\end{itemize}

Above items are related to this project execution period which is aligned to the work done at my job position at the Audiovisual Unit of the i2CAT Foundation. Concretely, this project is one of the next steps related to the LiveMediaStreamer framework, an open-source software developed by my team. One of this main functionalities is the capability of working as a software-based audio and video mixer, and this is the scenario that is going to be treated as the main service to be analysed.

Finally, in order to accomplish the main goal of this project, which is to offer improvements for real-time media production over cloud infrastructures, the topics above introduced are going to be studied together with the aim to propose and develop a platform prototype. Moreover, this platform prototype will be deployed in a laboratory environment in order to demonstrate and validate the proposal.







