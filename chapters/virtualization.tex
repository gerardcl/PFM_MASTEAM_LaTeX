\chapter{Virtualization}\label{D:virtualization}

Virtualization chapter focuses on the implementations done in order to prepare generic solutions for a media production platform prototype with the tools already introduced. This is preparing containers for each generic microservice:

\begin{itemize}
\item The core container with a LiveMediaStreamer instance already deployed inside and ready to use.
\item The HTTP REST API container with the middleware interface inside and ready to use
\item Both last items with Collectd inside
\end{itemize}

Therefore, let's introduce how Docker is prepared. First of all, the host operating system is an Ubuntu 14.04 LTS (--REFERENCE--), which is the Linux distribution version where the LMS and its related environment are developed. 

A



\begin{verbatim}
\begin{table}[htb]
\begin{center}
\begin{tabular}{|c|l|r|}
\hline
{\bf Títol de la Columna 1} & {\bf Títol de la Columna 2} & 
{\bf Títol de la Columna 3}  \\ \hline \hline
centrada        & a l'esquerra    & a la dreta       \\ \hline
centrada        & a l'esquerra    & a la dreta       \\ \hline
centrada        & a l'esquerra    & a la dreta       \\ \hline
centrada        & a l'esquerra    & a la dreta       \\ \hline
centrada        & a l'esquerra    & a la dreta       \\ \hline
\end{tabular}
\caption{Exemple de taula}
\label{T:prova}
\end{center}
\end{table}
\end{verbatim}




CONCLUSIONS CHAPTER

comentar que docker és com un git però de containers, on a dins els containers hi pots tenir un git també hehe

explicar que es prepara cada container de tal forma que sigui collectd ready (potser hi hauran dos tipus, collectdDocker o collectd dins container -multiple processes-)

