\chapter{Specific application metric methods}\label{ANX:appALG}

This appendix is showing concrete algorithm of interest in order to help understanding how application methods work.

\section{Input network metric method}\label{inmm}
\begin{quote}
\begin{verbatim}
void SCSSubsessionStats::
			periodicStatMeasurement(struct timeval const& timeNow) 
{
    unsigned secsDiff = timeNow.tv_sec - measurementEndTime.tv_sec;
    int usecsDiff = timeNow.tv_usec - measurementEndTime.tv_usec;
    double timeDiff = secsDiff + usecsDiff/1000000.0;
    measurementEndTime = timeNow;

    RTPReceptionStatsDB::Iterator statsIter(fSource->receptionStatsDB());

    RTPReceptionStats* stats = statsIter.next(True);
    if (stats != NULL) {
        double kBytesTotalNow = stats->totNumKBytesReceived();
        double kBytesDeltaNow = kBytesTotalNow - kBytesTotal;
        kBytesTotal = kBytesTotalNow;

        double kbpsNow = timeDiff == 0.0 ? 0.0 : 8*kBytesDeltaNow/timeDiff;
        if (kbpsNow < 0.0) kbpsNow = 0.0; // in case of roundoff error
        if (kbpsNow < kbitsPerSecondMin) kbitsPerSecondMin = kbpsNow;
        if (kbpsNow > kbitsPerSecondMax) kbitsPerSecondMax = kbpsNow;

        unsigned totReceivedNow = stats->totNumPacketsReceived();
        unsigned totExpectedNow = stats->totNumPacketsExpected();
        unsigned deltaReceivedNow = totReceivedNow - totNumPacketsReceived;
        unsigned deltaExpectedNow = totExpectedNow - totNumPacketsExpected;
        totNumPacketsReceived = totReceivedNow;
        totNumPacketsExpected = totExpectedNow;

        double lossFractionNow = deltaExpectedNow == 0 ? 0.0 : 1.0 -
        							 deltaReceivedNow/(double)deltaExpectedNow;

        if (lossFractionNow < packetLossFractionMin) {
            packetLossFractionMin = lossFractionNow;
        }
        if (lossFractionNow > packetLossFractionMax) {
            packetLossFractionMax = lossFractionNow;
        }

        minInterPacketGapUS = stats->minInterPacketGapUS();
        maxInterPacketGapUS = stats->maxInterPacketGapUS();
        totalGaps = stats->totalInterPacketGaps();
        jitter = stats->jitter();
    }
}
\end{verbatim}
\end{quote} 
\section{Output network metric method}\label{onmm}


\begin{quote}
\begin{verbatim}
void ConnRTCPInstance::
			periodicStatMeasurement(struct timeval const& timeNow) 
{
    unsigned currentNumBytes;
    double currentElapsedTime;

    RTPTransmissionStatsDB::Iterator 
    							statsIter(fSink->transmissionStatsDB());

    fSink->getTotalBitrate(currentNumBytes, currentElapsedTime);

    avgBitrate = currentElapsedTime == 0 ? 0.0 :
    					((8*currentNumBytes/currentElapsedTime)/1000.0);
    if(minBitrate > avgBitrate) minBitrate = avgBitrate;
    if(maxBitrate < avgBitrate) maxBitrate = avgBitrate;

    RTPTransmissionStats* stats;
    while((stats = statsIter.next()) != NULL){
        SSRC = stats->SSRC();
        packetLossRatio = stats->packetLossRatio();
        if(minPacketLossRatio > packetLossRatio) 
        							minPacketLossRatio = packetLossRatio;
        if(maxPacketLossRatio < packetLossRatio) 
        							maxPacketLossRatio = packetLossRatio;        
        
        roundTripDelay = stats->roundTripDelay();
        if(minRoundTripDelay > roundTripDelay) 
        							minRoundTripDelay = roundTripDelay;
        if(maxRoundTripDelay < roundTripDelay) 
        							maxRoundTripDelay = roundTripDelay;

        jitter = stats->jitter();
        if(minJitter > jitter) minJitter = jitter;
        if(maxJitter < jitter) maxJitter = jitter;
    }
}
\end{verbatim}
\end{quote} 
\section{Pipiline delay metric method}\label{pdmm}


\begin{quote}
\begin{verbatim}
void Reader::measureDelay()
{
    if(lastTs.count() < 0){
        lastTs = frame->getPresentationTime();
    }

    if (lastTs == frame->getPresentationTime()) {
        return;
    }

    timeCounter += frame->getPresentationTime() - lastTs;
    lastTs = frame->getPresentationTime();

    if(timeCounter >= windowDelay){
        avgDelay = delay / frameCounter;
        timeCounter = std::chrono::microseconds(0);
        delay = std::chrono::microseconds(0);
        frameCounter = 0;
    }
    
    delay += std::chrono::duration_cast<std::chrono::
    						microseconds>(std::chrono::system_clock::now() 
    											- frame->getOriginTime());
    frameCounter++;
}
\end{verbatim}
\end{quote} 

