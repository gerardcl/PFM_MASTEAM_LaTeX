\chapter{LMS HTTP RESTful API}\label{ANX:RESTAPI}

Proposed HTTP RESTful API's structure for managing an instance of the LiveMediaStreamer:

\begin{itemize}
\item Generic management queries
\begin{itemize}
\item Connect \hfill

Checks if an existing instance of LMS is running and sets the LMS port and LMS host to work with.
\begin{quote}
\begin{verbatim}
POST http://<host>:<port>/api/connect
JSON    {
            "port":<lms-port>,
            "host":"<lms-host>"
        }
\end{verbatim}
\end{quote}
\item Disconnect \hfill

Resets the running LMS instance and sets \verb|lms-port| and \verb|lms-host| to null in order to connect again to the same or any another LMS instance running.
\begin{quote}
\begin{verbatim}
GET http://<host>:<port>/api/disconnect
\end{verbatim}
\end{quote}
\item State \hfill

Gets the state object of the current LMS instance connected to (JSON object with the configured filters and paths).
\begin{quote}
\begin{verbatim}
GET http://<host>:<port>/api/state
\end{verbatim}
\end{quote}
\item Create a filter \hfill

Creates a filter (current types are: receiver, transmitter, demuxer, dasher, audioDecoder, audioEncoder, videoDecoder, videoResampler, videoEncoder, audioMixer, videoMixer) with an unique identifier.
\begin{quote}
\begin{verbatim}
POST http://<host>:<port>/api/createFilter
JSON    {
            "id": filterID,
            "type": "type"
        }
\end{verbatim}
\end{quote}        
\item Create a path of filters \hfill

Create a path of filters. A path can be a master path or an slave one, as shown:

\begin{itemize}

\item Master path
\begin{quote}
\begin{verbatim}
POST http://<host>:<port>/api/createPath
JSON    { 
            'id' : pathId, 
            'orgFilterId' : orgFilterId, 
            'dstFilterId' : dstFilterId, 
            'orgWriterId' : orgWriterId, 
            'dstReaderId' : dstReaderId, 
            'midFiltersIds' : [filterID1, filterID2,...] 
        }
\end{verbatim}
\end{quote}
\item Slave path of previous master path
\begin{quote}
\begin{verbatim}
POST http://<host>:<port>/api/createPath
JSON    { 
            'id' : pathId, 
            'orgFilterId' : filterID1, 
            'dstFilterId' : dstFilterId2, 
            'orgWriterId' : -1, 
            'dstReaderId' : dstReaderId2, 
            'midFiltersIds' : [filterID3, filterID4,...] 
        }  
\end{verbatim}
\end{quote}
\end{itemize}
\end{itemize}
\item Specific filter management queries        
\begin{itemize}
\item Configure an existing filter \hfill

Configures an existing filter (each filter has its own actions defined, check APPENDIX \ref{ANX:lmsarchfull})
\begin{quote}
\begin{verbatim}
PUT http://<host>:<port>/api/filter/:filterID
JSON    [
            {
                "action":"the action",
                    "params":{
                        "param1":param1,
                        "param2":"param2",
                        "param3":"param3",
                        "param4":true
                }
            }
        ]
\end{verbatim}
\end{quote}

\end{itemize}
\end{itemize}

So, with previous API proposal, the whole LMS's TCP socket API becomes simplified.

Moreover, specific responses format to client has been proposed as shown:

\begin{itemize}
\item Success messages \hfill

It may be a string, bool, array or another object, depending on the request method

\begin{quote}
\begin{verbatim}
JSON    {
            "message": { the incoming message }
        }
\end{verbatim}
\end{quote}        
\item Error messages \hfill

\begin{quote}
\begin{verbatim}
JSON    {
            "error": "the error message"
        }
\end{verbatim}
\end{quote}
\end{itemize}