\chapter{Exchanged e-mails with Live555 developer mailing list}\label{ANX:emailRoss}

E-mail sent:

\begin{quote}
\begin{verbatim}
Sender: Gerard Castillo Lasheras <gerard.castillo@i2cat.net>
Receiver: LIVE555 Streaming Media - development & 
									use <live-devel@ns.live555.com> 
Hi Ross,

I'm implementing statistics on our software (liveMediaStreamer framework) 
and I'd like to have access to the RTPTransmissionStatsDB. But, I do not
see how to get the RTPSink object (which has the RTPTransmissionStatsDB 
and its stats).

Which should be the proper way to get the RTPSink object related to my
OnDemandServerMediaSubsession childs? I've seen that 
OnDemandServerMediaSubsession has a friend classe StreamState 
which has the RTPSink associated but, anyway, 
I'm not able to have access to it.

Thanks in advance,

Kind regards,
\end{verbatim}
\end{quote} 

The reply:
\begin{quote}
\begin{verbatim}
Sender: Ross Finlayson <finlayson@live555.com>
Receiver: LIVE555 Streaming Media - development & 
									use <live-devel@ns.live555.com> 
									
First of all, note that while a "OnDemandServerMediaSubsession" 
object refers to a track of streamable media, a "RTPSink" object 
refers to a receiving client (or possibly multiple clients if 
"reuseFirstSource" is True).  So there’s (in general) a 
one-to-many relationship between "OnDemandServerMediaSubsession" 
and "RTPSink".  Thus, it doesn’t make sense to talk about 
*the* RTPSink object for your "OnDemandServerMediaSubsession".

However…
There are at least two possible ways to get access to the 
"RTPSink" objects:

1/ Note the pure virtual function "createNewRTPSink()" that you 
have implemented in your "OnDemandServerMediaSubsession" subclass.  
You can use your implementation of this function to get access 
to the "RTPTransmissionStatsDB" for the new "RTPSink", after 
you’ve created it.
The drawback of this approach, though, is that you don’t know when 
the "RTPSink" object later gets deleted, so - if you’re not 
careful - you may end up holding a reference or pointer to a 
"RTPTransmissionStatsDB" that has been deleted.

2/ Define a subclass "myRTCPInstance" of the "RTCPInstance" 
class.  Then, in your "OnDemandServerMediaSubsession" subclass, 
reimplement the "createRTCP()" virtual function to create a 
"myRTCPInstance" object, rather than a "RTCPInstance" 
object.  Note that "createRTCP()" contains a "sink" 
parameter, pointing to a "RTPSink", from which you can 
get the "RTPTransmissionStatsDB".
The advantage of this approach over approach 1/ is that 
- by defining a subclass of "RTCPInstance", you can learn when 
the "RTPInstance" object gets deleted, and thus when the 
"RTPSink" object gets deleted.  (The "RTCPInstance" object 
always gets deleted immediately before the "RTPSink" 
object.)  Thus, you can use your "myRTCPInstance" destructor 
to figure out when the "RTPTransmissionStatsDB" should no 
longer be used.
									
\end{verbatim}
\end{quote} 
