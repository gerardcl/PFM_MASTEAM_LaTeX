\chapter{Docker, Nginx and Collectd configuration files}\label{ANX:dockerFiles}

This appendix is listing specific Docker, Nginx and Collectd configuration files.

\section{Basic LMS container}\label{ANX:dockerFiles1}

Next Docker file installs latest development commit into the container to be built and prepares the image to run LMS as a service.

\begin{verbatim}
# LiveMediaStreamer Container
FROM ubuntu:14.04
MAINTAINER Gerard CL <gerardcl@gmail.com>

RUN apt-get update && apt-get -y upgrade

RUN apt-get -y install git cmake autoconf automake build-essential \ 
libass-dev libtheora-dev libtool libvorbis-dev pkg-config zlib1g-dev \
libcppunit-dev yasm libx264-dev  libmp3lame-dev  libopus-dev \
libvpx-dev liblog4cplus-dev libtinyxml2-dev opencv-data \
libopencv-dev mercurial cmake-curses-gui vim libcurl3 wget curl 

RUN adduser --disabled-password --gecos '' lms && adduser lms sudo \
	&& echo '%sudo ALL=(ALL) NOPASSWD:ALL' >> /etc/sudoers

USER lms

RUN hg clone https://bitbucket.org/multicoreware/x265 /home/lms/x265 \
	&& cd /home/lms/x265 && cmake -G "Unix Makefiles" ./source \
	&& make -j && sudo make install && sudo ldconfig

RUN git clone https://github.com/mstorsjo/fdk-aac.git/ /home/lms/fdk-aac \
	&& cd /home/lms/fdk-aac && libtoolize && ./autogen.sh \
	&& ./configure && make -j && sudo make install && sudo ldconfig

RUN cd /home/lms && wget http://ffmpeg.org/releases/ffmpeg-2.7.tar.bz2 \
	&& tar xjvf ffmpeg-2.7.tar.bz2 && cd ffmpeg-2.7 \
	&& ./configure --enable-gpl --enable-libass --enable-libtheora \
	--enable-libvorbis --enable-libx264 --enable-nonfree --enable-shared \
	--enable-libopus --enable-libmp3lame --enable-libvpx \
	--enable-libfdk_aac --enable-libx265 && make -j \
	&& sudo make install && sudo ldconfig

RUN cd /home/lms && wget \
	http://www.live555.com/liveMedia/public/live555-latest.tar.gz \
	&& tar xaf live555-latest.tar.gz && cd live \
	&& ./genMakefiles linux-with-shared-libraries && make -j \
	&& sudo make install && sudo ldconfig

RUN git clone https://github.com/ua-i2cat/livemediastreamer.git \
	/home/lms/livemediastreamer && cd /home/lms/livemediastreamer \
	&& git checkout development && ./autogen.sh \
	&& make -j && sudo make install && sudo ldconfig

EXPOSE 5000-5017/udp
EXPOSE 8554-8564
EXPOSE 7777

CMD ["/usr/local/bin/livemediastreamer","7777"] 
\end{verbatim}

\section{HTTP REST API container}\label{ANX:dockerFiles2}

Next Docker file installs Node.js and clones latest HTTP REST API for LMS from the GitHub repository. It also prepares the image in order to run the HTTP REST API middleware for the LMS framework.

\begin{verbatim}
# LiveMediaStreamer Container
FROM ubuntu:14.04

MAINTAINER Gerard CL <gerardcl@gmail.com>

RUN apt-get update && apt-get -y upgrade

RUN apt-get -y install git npm

RUN adduser --disabled-password --gecos '' lms \
&& adduser lms sudo \
&& echo '%sudo ALL=(ALL) NOPASSWD:ALL' >> /etc/sudoers

USER lms

RUN cd /home/lms \
&& git clone https://github.com/ua-i2cat/LMStoREST.git \
/home/lms/LMStoREST && cd /home/lms/LMStoREST && npm install

EXPOSE 8080
CMD ["nodejs", "/home/lms/LMStoREST/lms-middleware.js"]
\end{verbatim}

\section{Running multiple processes within a container}\label{ANX:dockerFiles3}

This Docker file installs Nginx and LMS inside the same container in order to be ready to serve through HTTP Nginx server the MPEG-DASH files generated for the LMS framework (it externally requires to be configured as a transcoder to MPEG-DASH). Then, superviord, which is also installed, runs as the default image process. Supervisord is managing each defined process to be executed.  

\begin{verbatim}
# LiveMediaStreamer Container
# and Nginx server for MPEG-DASH streaming
FROM ubuntu:14.04
MAINTAINER Gerard CL <gerardcl@gmail.com>

RUN apt-get update && apt-get -y upgrade

RUN apt-get -y install git cmake autoconf automake build-essential \ 
libass-dev libtheora-dev libtool libvorbis-dev pkg-config zlib1g-dev \
libcppunit-dev yasm libx264-dev  libmp3lame-dev  libopus-dev \
libvpx-dev liblog4cplus-dev libtinyxml2-dev opencv-data \
libopencv-dev mercurial cmake-curses-gui vim libcurl3 wget curl 

RUN apt-get -y install nginx supervisor
RUN mkdir -p /var/lock/nginx /var/run/nginx \
	/var/lock/livemediastreamer /var/run/livemediastreamer \
	/var/log/supervisor

ADD ./nginx.conf /etc/nginx/nginx.conf
ADD ./supervisord.conf /etc/supervisor/conf.d/supervisord.conf


RUN adduser --disabled-password --gecos '' lms \
&& adduser lms sudo \
&& echo '%sudo ALL=(ALL) NOPASSWD:ALL' >> /etc/sudoers

RUN mkdir -p /home/lms/dashSegments

USER lms

RUN hg clone https://bitbucket.org/multicoreware/x265 /home/lms/x265 \
	&& cd /home/lms/x265 && cmake -G "Unix Makefiles" ./source \
	&& make -j && sudo make install && sudo ldconfig

RUN git clone https://github.com/mstorsjo/fdk-aac.git/ /home/lms/fdk-aac \
	&& cd /home/lms/fdk-aac && libtoolize && ./autogen.sh \
	&& ./configure && make -j && sudo make install && sudo ldconfig

RUN cd /home/lms && wget http://ffmpeg.org/releases/ffmpeg-2.7.tar.bz2 \
	&& tar xjvf ffmpeg-2.7.tar.bz2 && cd ffmpeg-2.7 \
	&& ./configure --enable-gpl --enable-libass --enable-libtheora \
	--enable-libvorbis --enable-libx264 --enable-nonfree --enable-shared \
	--enable-libopus --enable-libmp3lame --enable-libvpx \
	--enable-libfdk_aac --enable-libx265 && make -j \
	&& sudo make install && sudo ldconfig

RUN cd /home/lms && wget \
	http://www.live555.com/liveMedia/public/live555-latest.tar.gz \
	&& tar xaf live555-latest.tar.gz && cd live \
	&& ./genMakefiles linux-with-shared-libraries && make -j \
	&& sudo make install && sudo ldconfig

RUN git clone https://github.com/ua-i2cat/livemediastreamer.git \
	/home/lms/livemediastreamer && cd /home/lms/livemediastreamer \
	&& git checkout development && ./autogen.sh \
	&& make -j && sudo make install && sudo ldconfig

USER root

EXPOSE 5000-5017/udp
EXPOSE 8554-8564
EXPOSE 7777
EXPOSE 8080

CMD ["/usr/bin/supervisord"] 
\end{verbatim}

\section{Nginx server file configuration example}\label{ANX:nginxexample}

This is a basic example to configure the Nginx server in order to server the files generated for the LMS. It is not recommended to define servers in the same nginx.conf file (servers should be defined through the sites available/enabled directories) but this is done that way in order to demonstrate and example a complete but basic Nginx server configuration.

\begin{verbatim}
# this sets the user nginx will run as, 
# and the number of worker processes
user nobody nogroup;
worker_processes  1;

# setup where nginx will log errors to 
# and where the nginx process id resides
error_log  /var/log/nginx/error.log;
pid        /var/run/nginx.pid;

events {
  worker_connections  1024;
  # set to on if you have more than 1 worker_processes 
  accept_mutex off;
}

http {
  include       /etc/nginx/mime.types;

  default_type application/octet-stream;
  access_log /tmp/nginx.access.log combined;
 
  # use the kernel sendfile
  sendfile        on;
  # prepend http headers before sendfile() 
  tcp_nopush     on;

  keepalive_timeout  5;
  tcp_nodelay        on;

  gzip  on;
  gzip_vary on;
  gzip_min_length 500;
  
  gzip_disable "MSIE [1-6]\.(?!.*SV1)";
  gzip_types text/plain text/xml text/css
     text/comma-separated-values
     text/javascript application/x-javascript
     application/atom+xml image/x-icon;

  # configure the virtual host
  server {
    # replace with your domain name
    server_name localhost;
    root /home/lms/dashSegments;
    # port to listen for requests on
    listen 8090;
    # maximum accepted body size of client request 
    client_max_body_size 4G;
    # the server will close connections after this time 
    keepalive_timeout 5;
    
    add_header Access-Control-Allow-Origin "*";
    add_header Access-Control-Allow-Methods "GET, OPTIONS";
    add_header Access-Control-Allow-Headers "origin, authorization, accept";
    add_header Cache-Control no-cache;
    
    location / {
        add_header Access-Control-Allow-Origin "*";
        add_header Access-Control-Allow-Methods "GET, OPTIONS";
        add_header Access-Control-Allow-Headers "origin, authorization, accept";
        add_header Cache-Control no-cache;
    }
  }
}

daemon off;
\end{verbatim}

\section{Sharing volumes within containers}\label{ANX:dockerFiles4}

This containers installs Nginx and runs it as default image command. This is an isolated Nginx server container which is serving the LMS files through the shared volumes between the basic LMS container and this one.

\begin{verbatim}
# Nginx Container - LMS special one
FROM ubuntu:14.04
MAINTAINER Gerard CL <gerardcl@gmail.com>

RUN apt-get update && apt-get install --fix-missing \
	&& apt-get -y upgrade

RUN apt-get -y install nginx 

ADD ./nginx.conf /etc/nginx/nginx.conf

RUN adduser --disabled-password --gecos '' lms \
	&& adduser lms sudo \
	&& echo '%sudo ALL=(ALL) NOPASSWD:ALL' >> /etc/sudoers

RUN mkdir -p /home/lms/dashSegments

EXPOSE 8090

CMD ["/usr/sbin/nginx"]  
\end{verbatim}

\section{Collectd client container}\label{ANX:dockerFiles5}

This is a basic Collectd client container build.

\begin{verbatim}

FROM    ubuntu:14.04
MAINTAINER Gerard CL <gerardcl@gmail.com>

ENV     DEBIAN_FRONTEND noninteractive

RUN apt-get update
RUN apt-get -y install collectd curl python-dev python-pip

ADD collectd.conf.tpl /etc/collectd/collectd.conf.tpl

RUN pip install envtpl
ADD start_container /usr/bin/start_container
RUN chmod +x /usr/bin/start_container
CMD start_container

\end{verbatim}

\section{Collectd client configuration}\label{ANX:collectdFiles1}

This is a basic Collectd client configuration template file.

\begin{verbatim}
Hostname "{{ LMS_NAME }}"
FQDNLookup true

Interval 1
Timeout 4
ReadThreads 5

LoadPlugin syslog
LoadPlugin cpu
LoadPlugin load
LoadPlugin memory
LoadPlugin network

<Plugin "syslog">
  LogLevel "info"
  NotifyLevel "OKAY"
</Plugin>

<Plugin network>
  Server "{{ GRAPHITE_HOST }}" "{{ GRAPHITE_PORT | default("25826") }}"
  ReportStats true
</Plugin>
\end{verbatim}

\section{Collectd server and Graphite container}\label{ANX:dockerFiles6}

This Docker file installs and configures all the dependences required to run a basic Graphite system which is served by a Collectd server which listens from external Collectd containerized clients.

\begin{verbatim}
# LiveMediaStreamer Container
FROM ubuntu:14.04
MAINTAINER Gerard CL <gerardcl@gmail.com>

RUN apt-get update
RUN apt-get install -y python-cairo collectd-core libgcrypt11 \
python-virtualenv build-essential python-dev supervisor sudo

RUN adduser --system --group --no-create-home collectd \
&& adduser --system --home /opt/graphite graphite

RUN sudo -u graphite virtualenv --system-site-packages ~graphite/env

RUN echo "django \n \
  python-memcached \n \
  django-tagging \n \
  twisted \n \
  gunicorn \n \
  whisper \n \
  carbon \n \
  graphite-web" > /tmp/graphite_reqs.txt

RUN sudo -u graphite HOME=/opt/graphite /bin/sh -c ". \
~/env/bin/activate && pip install -r /tmp/graphite_reqs.txt"

ADD collectd/collectd.conf /etc/collectd/
ADD supervisor/ /etc/supervisor/conf.d/
ADD graphite/settings.py /opt/graphite/webapp/graphite/
ADD graphite/local_settings.py /opt/graphite/webapp/graphite/
ADD graphite/mkadmin.py /opt/graphite/webapp/graphite/
ADD graphite/storage-schemas.conf /opt/graphite/conf/

RUN cp /opt/graphite/conf/carbon.conf.example \
/opt/graphite/conf/carbon.conf
RUN cp /opt/graphite/conf/graphite.wsgi.example \
/opt/graphite/webapp/graphite/graphite_wsgi.py
RUN cp /opt/graphite/conf/graphite.wsgi.example \
/opt/graphite/conf/graphite.wsgi
RUN cp /opt/graphite/conf/storage-aggregation.conf.example \
/opt/graphite/conf/storage-aggregation.conf

RUN sed -i "s#^\(SECRET_KEY = \).*#\1\"`python -c 'import os; import base64; \
print(base64.b64encode(os.urandom(40)))'`\"#" \
/opt/graphite/webapp/graphite/app_settings.py
RUN sudo -u graphite HOME=/opt/graphite PYTHONPATH=/opt/graphite/lib/ \
/bin/sh -c "cd ~/webapp/graphite && ~/env/bin/python manage.py syncdb --noinput"
RUN sudo -u graphite HOME=/opt/graphite PYTHONPATH=/opt/graphite/lib/ \
/bin/sh -c "cd ~/webapp/graphite && ~/env/bin/python mkadmin.py"

EXPOSE 8080 25826/udp

CMD exec supervisord -n
\end{verbatim}

\section{Collectd server configuration}\label{ANX:collectdFiles2}

This is a Collectd server configuration example.

\begin{verbatim}
Hostname "localhost"
FQDNLookup true
Interval 1

LoadPlugin syslog
LoadPlugin network
LoadPlugin write_graphite

<Plugin syslog>
	LogLevel "info"
	NotifyLevel "OKAY"
</Plugin>

<Plugin network>
	Listen "*" "25826"
	ReportStats true
</Plugin>

<Plugin write_graphite>
	<Node "graphing">
		Host "localhost"
		Port "2003"
		Protocol "tcp"
		LogSendErrors true
		Prefix "collectd."
		StoreRates true
		AlwaysAppendDS false
		EscapeCharacter "_"
	</Node>
</Plugin>
\end{verbatim}

\section{Collectd tail plugin example configuration}\label{ANX:collectdTailFiles2}

Next tail plugin configuration is performing specific regular expression matching to bind logged metrics from the LMS demo script. Each logged line to stdout is formatted as shown next:

\begin{verbatim}
|PATHID|value|avgDelay-5004|value|lostBlocs-5004|value|PATHID|value
|avgDelay-5006|value|lostBlocs-5006|value|RXmediaType|video
|videoRXavgBitRateInKbps|value|videoRXavgPacketLossPercentage
|value|videoRXavgInterPacketGapInMiliseconds|value
|videoRXcurJitterInMicroseconds|value|RXmediaType|audio
|audioRXavgBitRateInKbps|value|audioRXavgPacketLossPercentage
|value|audioRXavgInterPacketGapInMiliseconds|value
|audioRXcurJitterInMicroseconds|value|TX_URI|value
|TX_SSRC-1|value|TXavgBitrateInKbps-1|value
|TXpacketLossRatio-1|value|TXjitterInMicroseconds-1
|value|TXroundTripDelayMilliseconds-1|value|TX_SSRC-2
|value|TXavgBitrateInKbps-2|value|TXpacketLossRatio-2
|value|TXjitterInMicroseconds-2|value|TXroundTripDelayMilliseconds-2|value|
\end{verbatim}

Then, each new line is parsed through the tail plugin of the Collectd configuration. An example is shown next:
\begin{verbatim}
<Plugin tail>
<File "/home/lms/logs/lms.log">
Instance "lmsTest"
<Match>
Regex ".*avgDelay-5004\\|([0-9]*).*"
DSType "CounterAdd"
Type counter
Instance "VideoAVGdelay-us"
</Match>
<Match>
Regex ".*lostBlocs-5004\\|([0-9]*).*"
DSType "CounterAdd"
Type counter
Instance "VideoLostBlocs"
</Match>
<Match>
Regex ".*avgDelay-5006\\|([0-9]*).*"
DSType "CounterAdd"
Type counter
Instance "AudioAVGdelay-us"
</Match>
<Match>
Regex ".*lostBlocs-5006\\|([0-9]*).*"
DSType "CounterAdd"
Type counter
Instance "AudioLostBlocs"
</Match>
<Match>
Regex ".*RXmediaType\\|video.*videoRXavgBitRateInKbps\\|([0-9]*).*"
DSType "CounterAdd"
Type counter
Instance "videoRXavgBitRateInKbps"
</Match>
<Match>
Regex ".*RXmediaType\\|video.*videoRXavgPacketLossPercentage\\|([0-9]*).*"
DSType "CounterAdd"
Type counter
Instance "videoRXavgPacketLossPercentage"
</Match>
<Match>
Regex ".*RXmediaType\\|video.*videoRXavgInterPacketGapInMiliseconds\\|([0-9]*).*"
DSType "CounterAdd"
Type counter
Instance "videoRXavgInterPacketGapInMiliseconds"
</Match>
<Match>
Regex ".*RXmediaType\\|video.*videoRXcurJitterInMicroseconds\\|([0-9]*).*"
DSType "CounterAdd"
Type counter
Instance "videoRXcurJitterInMicroseconds"
</Match>
<Match>
Regex ".*RXmediaType\\|audio.*audioRXavgBitRateInKbps\\|([0-9]*).*"
DSType "CounterAdd"
Type counter
Instance "audioRXavgBitRateInKbps"
</Match>
<Match>
Regex ".*RXmediaType\\|audio.*audioRXavgPacketLossPercentage\\|([0-9]*).*"
DSType "CounterAdd"
Type counter
Instance "audioRXavgPacketLossPercentage"
</Match>
<Match>
Regex ".*RXmediaType\\|audio.*audioRXavgInterPacketGapInMiliseconds\\|([0-9]*).*"
DSType "CounterAdd"
Type counter
Instance "audioRXavgInterPacketGapInMiliseconds"
</Match>
<Match>
Regex ".*RXmediaType\\|audio.*audioRXcurJitterInMicroseconds\\|([0-9]*).*"
DSType "CounterAdd"
Type counter
Instance "audioRXcurJitterInMicroseconds"
</Match>
<Match>
Regex ".*TXavgBitrateInKbps-1\\|([0-9]*).*"
DSType "CounterAdd"
Type counter
Instance "VideoTXavgBitrateInKbps"
</Match>
<Match>
Regex ".*TXpacketLossRatio-1\\|([0-9]*).*"
DSType "CounterAdd"
Type counter
Instance "VideoTXpacketLossRatio"
</Match>
<Match>
Regex ".*TXjitterInMicroseconds-1\\|([0-9]*).*"
DSType "CounterAdd"
Type counter
Instance "VideoTXjitterInMicroseconds"
</Match>
<Match>
Regex ".*TXroundTripDelayMilliseconds-1\\|([0-9]*).*"
DSType "CounterAdd"
Type counter
Instance "VideoTXroundTripDelayMilliseconds"
</Match>
<Match>
Regex ".*TXavgBitrateInKbps-2\\|([0-9]*).*"
DSType "CounterAdd"
Type counter
Instance "AudioTXavgBitrateInKbps"
</Match>
<Match>
Regex ".*TXpacketLossRatio-2\\|([0-9]*).*"
DSType "CounterAdd"
Type counter
Instance "AudioTXpacketLossRatio"
</Match>
<Match>
Regex ".*TXjitterInMicroseconds-2\\|([0-9]*).*"
DSType "CounterAdd"
Type counter
Instance "AudioTXjitterInMicroseconds"
</Match>
<Match>
Regex ".*TXroundTripDelayMilliseconds-2\\|([0-9]*).*"
DSType "CounterAdd"
Type counter
Instance "AudioTXroundTripDelayMilliseconds"
</Match>
</File>
</Plugin>
\end{verbatim}
