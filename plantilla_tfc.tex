%%%%%%%%%%%%%%%%%%%%%%%%%%%%%%%%%%%%%%%%%%%%%%%%%%%%%%%%%%%%%%%%%%%%%%%%%%%%%
%%%%%%                                                                  %%%%% 
%%%%%%          Maqueta de memòria TFC/PFC de l'EETAC                   %%%%% 
%%%%%%                                                                  %%%%% 
%%%%%%%%%%%%%%%%%%%%%%%%%%%%%%%%%%%%%%%%%%%%%%%%%%%%%%%%%%%%%%%%%%%%%%%%%%%%%
%%%%%%%%%%%%%%%%%%%%%%%%%%%%%%%%%%%%%%%%%%%%%%%%%%%%%%%%%%%%%%%%%%%%%%%%%%%%%
%%                                                                         %%
%%          Autor: Xavier Prats i Menéndez (xavier.prats@upc.edu)          %% 
%%                  Technical University of Catalonia (UPC)                %%
%%                                                                         %%
%%%%%%%%%%%%%%%%%%%%%%%%%%%%%%%%%%%%%%%%%%%%%%%%%%%%%%%%%%%%%%%%%%%%%%%%%%%%%
%%      This work is licensed under the Creative Commons  Attribution-     %%
%%   -Noncommercial-Share Alike 3.0 Spain License. To view a copy of this  %% 
%%    license, visit http://creativecommons.org/licenses/by-nc-sa/3.0/es/  %%
%%    or send a letter to Creative Commons, 171 Second Street, Suite 300,  %%
%%                  San Francisco,California, 94105, USA.                  %%
%%%%%%%%%%%%%%%%%%%%%%%%%%%%%%%%%%%%%%%%%%%%%%%%%%%%%%%%%%%%%%%%%%%%%%%%%%%%%
%% Versió 2.1 - Juliol 2012                                                %%
%%%%%%%%%%%%%%%%%%%%%%%%%%%%%%%%%%%%%%%%%%%%%%%%%%%%%%%%%%%%%%%%%%%%%%%%%%%%%

%%% NOTA: els seguents packages son necessaris per utilitzar la
%%%       plantilla seguent:
%%%       ifthen,calc,helvet,pslatex,fancyhdr,nextpage,subfigure,tocloft,graphicx,url

%%% NOTA: Es possible que algunes distribuicions Linux o Windows.
%%%       no portin aquests paquets instal·lats per defecte.
%%%       En aquest cas els haureu d'instal·lar manualment.


%%%%%%%%%%%%%%%%%%%%%%%%%%%%%%%%%%%%%%%%%%%%%%%%%%%%%%%%%%%%%%%%%%%%%%%%%%%%%
% 1- INICIALITZACIÓ
%%%%%%%%%%%%%%%%%%%%%%%%%%%%%%%%%%%%%%%%%%%%%%%%%%%%%%%%%%%%%%%%%%%%%%%%%%%%%

\documentclass[english,final]{setup/eetac_tfc_pfc}
%% * OPCIONS A CONFIGURAR al \documentclass
%%    - Estat del document: final o draft
%%      NOTA: Draft no inserta les figures i marca només l'espai que
%%      ocupen. També s'indica quan el text sobrepassa els marges.
%%      Draft és molt útil per compilar ràpid el document si no és important
%%      en aquell moment visualitzar les figures.
%%    - Idioma PRINCIPAL del document: catalan, spanish, english, french...

\usepackage[english]{babel}
%%  * INCLOURE TOTS ELS IDIOMES QUE S'USARAN EN EL DOCUMENT
%%    NOTA: per canviar d'idioma al mig del document usar:
%%          \selectlanguage{nom_idioma}
%%%%%%%%%%%%%%%%%%%%%%%%%%%%%%%%%%%%%%%%%%%%%%%%%%%%%%%%%%%%%%%%%%%%%%%%%%%%%

%%%%%%%%%%%%%%%%%%%%%%%%%%%%%%%%%%%%%%%%%%%%%%%%%%%%%%%%%%%%%%%%%%%%%%%%%%%%%
% 2- CÀRREGA DE PAQUETS ADICIONALS (OPCIONALS)
%%%%%%%%%%%%%%%%%%%%%%%%%%%%%%%%%%%%%%%%%%%%%%%%%%%%%%%%%%%%%%%%%%%%%%%%%%%%%

%%% NOTA: Es possible que algunes distribuicions Linux o Windows.
%%%       no portin aquests paquets instal·lats per defecte.
%%%       En aquest cas els haureu d'instal·lar manualment.

%% El paquet inputenc és extramadament útil. 
%% Permet escriure els accents directament amb l'editor de texte
%% sense haver de fer coses com per exemple: introducci\'o
%% Heu d'especificar la codificació de caracters que utilitzeu pel
%% vostre fitxer (en aquest exemple utf8)
\usepackage[utf8]{inputenc}

%% Símbols matemàtics de la American Mathematical Society
\usepackage{amssymb,amsmath, amsfonts}  

%% El paquet array proporciona eines molt útils a l'hora de fer 
%% equacions amb matrius
\usepackage{array}             

%% Paquet que permet fer taules fusionant cel·les de files consecutives
\usepackage{multirow}          

%% Paquet molt útil en cas de tenir taules molt llargues que 
%   ocupin vàries pàgines
\usepackage{longtable}          

%% Permet canviar els colors del document
%\usepackage{color,colortbl}

%% Paquet molt útil que permet activar links en el PDF final.
%% * NO OBLIDAR DE CONFIGURAR els quatre primer camps!
\usepackage[
  pdfauthor={Gerard Castillo Lasheras},            % Configurar adientment
  %pdftitle={Treball Fi de Carrera - autor}, % Configurar adientment
  %pdfsubject={Titol del TFC aqui},          % Configurar adientment
  % Modificació respecte a la versió 2.1 - Iván Padilla Montero - Juliol 2014
  pdftitle={Treball Fi de Màster - Gerard Castillo Lasheras}, % Configurar adientment
  pdfsubject={Study and proposal of a distributed and scalable real-time media production platform (SaaS&P)},          % Configurar adientment  
  pdfkeywords={media, audiovisual, content, cloud, real-time, distributed, virtualization},    % Configurar adientment
  pdfcreator={EETAC-UPC}, 
  pdfproducer={LaTeX, dvipdf},
  pdfdisplaydoctitle=true, plainpages=false, linktocpage=true,         
  colorlinks=true, linkcolor=blue,citecolor=blue,urlcolor=blue,
  hyperfootnotes=false, pagebackref=true, pdfpagelabels=true,
  pdfpagemode=UseOutlines,
]{hyperref} 

%% NOTA IMPORTANT!:
%% Per tal que hyperef funcioni correctament amb els capitols o seccions no
%% numerats (\chapter*{}), com per exemple introducció, conclusions i bibliografia
%% cal posar les dues comandes seguents ABANS del \chapter*{} en questió
%\cleardoublepage
%\phantomsection

%% Permet trencar links URL. 
%% Atenció! afegir aquest paquet DESPRES del hyperref!!
\usepackage{breakurl} 

%% Permet arranjar matricialment multiples figures
%% NOTA: afegir aquest paquet DESPRES del hyperref!!
%%       Si no es desitja utilitzar aquest paquet, comentar la linia seguent
%%       i anar TAMBE al fitxer de classe (eetac_tfc_pfc.cls) per substituir: 
%%       \RequirePackage[subfigure]{tocloft}  per  \RequirePackage{tocloft}
\usepackage{subfigmat}         

%%%%%%%%%%%%%%%%%%%%%%%%%%%%%%%%%%%%%%%%%%%%%%%%%%%%%%%%%%%%%%%%%%%%%%%%%%%%%


%%%%%%%%%%%%%%%%%%%%%%%%%%%%%%%%%%%%%%%%%%%%%%%%%%%%%%%%%%%%%%%%%%%%%%%%%%%%%
% 3- DOCUMENT
%%%%%%%%%%%%%%%%%%%%%%%%%%%%%%%%%%%%%%%%%%%%%%%%%%%%%%%%%%%%%%%%%%%%%%%%%%%%%

%%% Configuració de les dades i variables boleanes rellevants del document:
\input{setup/dades.tex}  

%%% Configuració de MACROS o ENTORNS (opcionals) definides per l'usuari:
\input{setup/user-macros.tex}  

%%% Configuració manual de les regles d'hyphenation:
\input{setup/hyphenation.tex}  

\begin{document}

%% Seleccionar l'idioma principal del document:
\selectlanguage{english}

\beforepreface  

%% RESUM i OVERVIEW
%%%%%%%%%%%%%%%%%%%%%%%%%%%%%%%%%%%%%%%%%%%%%%%%%%%%%%%%%%%%%%%%%%%%%%%%%%%%%
% NOTA: les longituds passades com a parametres d'entrada  s'han
%        d'ajustar manualment fins que el requadre del resum/overview
%        ocupi tota la pàgina. 

%%% Resum en català (o castellà)
\selectlanguage{english}   
\begin{resum}{10cm}
  La indústria de producció de continguts audiovisuals (e.g.: cadenes de radiodifusió, productores de baix pressupost, \ldots) ha estat, i encara està, emprant tecnologies rígides i difícils d'escalar per al transport i gestió dels seus fluxos a través de les seves cadenes de producció. Però, des de principis de l'any 2000, ha estat duent-se a terme una adopció gradual de les tecnologies IP.
  \\
  \\
  A més a més, la majoria d'aquestes tecnologies impliquen grans costos de desplegament i manteniment (e.g.: maquinari específic, cablejat específic i costós, \ldots). Per aquest motiu, es proposa l'estudi de tecnologies IP i, concretament, tecnologies relacionades amb el concepte de la computació distribuïda, i al núvol, per tal de proposar solucions per a abaratir els costos i millorar les possibilitats de producció de continguts audiovisuals.
  \\
  \\
  Concretament, aquest projecte s'enfoca en analitzar, proposar, desenvolupar i demostrar tecnologies específiques de virtualització, monitoratge i aplicació, que ofereixen solucions a les qüestions esmentades.
  
\end{resum}

%%% Resum en anglès
\selectlanguage{english}   
\begin{overview}{11cm}
  The audio-visual media content production industry (e.g.: broadcasters, small production companies, \ldots) has been, and already is, employing rigid and difficult to scale technologies to transport and manage their streams through their processing chain. But, since early 2000s, a gradually adoption of IP technologies has been happening.
  \\
  \\
  Furthermore, most of these technologies involve large deployment and maintenance costs (e.g.: specific hardware, specific and costly wiring, \ ldots). For this reason, it is proposed the study of IP technology, specifically technology related to the distributed cloud computing concept, in order to propose solutions to reduce costs and improve the chances of producing audiovisual content.
  \\
  \\
  Specifically, this project focuses on analyse, propose, develop and demonstrate specific virtualization, monitoring and application technologies, which provide solutions to these mentioned issues.     

\end{overview}

% Tornar a l'idioma principal del document
\selectlanguage{english}  

%NOTA: En cas d'utilitzar l'espanyol com a idioma principal del document, el
%      latex anomena les taules com a 'Cuadros'. Si es desitja canviar aquesta
%      nomenclatura i utilitzar la paraula 'Tabla' descomentar les línies següents:
%\def\listtablename{Índice de tablas}
%\def\tablename{Tabla}%



% Amb aqueta comanda indiquem que ja s'han inclòs tots els apartats del prefaci del 
% projecte o podem començar a incloure els capitols de la memòria
\afterpreface


%%%%%%%%%%%%%%%%%%%%%%%%%%%%%%%%%%%%%%%%%%%%%%%%%%%%%%%%%%%%%%%%%%%%%%%%%%
%%%%%% INCLOURE A PARTIR D'AQUÍ TOTS ELS CAPÍTOLS DE LA MEMORIA   %%%%%%%%
%%%%%%%%%%%%%%%%%%%%%%%%%%%%%%%%%%%%%%%%%%%%%%%%%%%%%%%%%%%%%%%%%%%%%%%%%%

% NOTA: recordar que la introducció i les conclusions són capítols NO
%       enumerats, per tant s'ha d'usar \chapter*

% NOTA: és aconsellable incloure els capítols de la memòria en fitxers 
%       separats utlitzant la comanda \input  Per exemple:
%       \input{capitol1}  
%       que farà que s'inclogui el fitxer capitol1.tex

% NOTA: Si es vol incloure agraïments i/o glosari, fer-ho utilitzant 
% \chapter*{} i incloure'ls abans la introducció

\cleardoublepage
\phantomsection
\chapter*{Introduction}
 
The audio-visual media content production industry (e.g.: broadcasters, small production companies, \ldots) has been, and already is, employing rigid and difficult to scale technologies to transport and manage their streams through their processing chain. But, since early 2000s, a gradually adoption of IP technologies has been happening. Key examples of this adoption are that TV content media production is already digital based (i.e.: broadcasting channels), its media transport layer is circuit oriented. Moreover, IP networks offer enhancements over operational and cost issues and it is the next step to the audiovisual media production.

Since few years ago, the broadcasting industry is pushing on to adopt IP as the transport technology because of several benefits such as:

\begin{itemize}
  \item Enhanced agility and flexibility of the broadcast workflows
  \item Convergence of services (i.e.: audio, video, metadata and generic data)
  \item Format agnosticism to support the adoption of coming new UHDTV formats such as 8K 
  \item Economy of scale by integrating broadcasting industry into the far more massive IT industry
\end{itemize}

But, there is still a lot of work to be done to achieve complete IP convergence, and lots of new possibilities thanks to different architectures and configurations that can be applied on OTT (Over-The-Top \cite{ottVSiptv}) content management systems. 

Almost all post-production environments has become file-based and can now be completely migrated to IP infrastructures. Main examples are some control and management services, and media content distribution over Internet (i.e.: live or on demmand media streaming to end-users) or over proprietary networks (i.e.: IPTV \cite{ottVSiptv}).

But, the live production environment has yet to be migrated to IP. Main reasons are due to some stringent constraints such as levels of synchronization, extremely low packet loss levels, high-bandwidth demand or jitter variation because these are not all intrinsically assured by current IP technologies.

Therefore, in order to accomplish the main goal of this project, which is to offer improvements for real-time media content production over cloud infrastructures, the topics which are introduced next are going to be studied together with the aim to propose and develop a platform prototype. Moreover, this platform prototype will be deployed in an experimental environment in order to demonstrate and validate the proposal.

Although the huge amount of issues to be solved, this project is focused on few issues related to cloud migration for real-time media production at application plane. Specifically, these are:

\begin{itemize}
\item Virtualization layer \hfill 

The virtualization paradigm offers specific solutions to improve scalability, robustness and reliability of any platform and services to be deployed over a cloud environment (n-tier applications development \cite{n-tier architecture}). 

\item Monitoring layer \hfill 

The monitoring layer is a crucial tool to be deployed which offers the capability of getting non-stop feedback from deployed services in order to actuate in real-time over any defined alarm. This layer gives also the chance to find and solve infrastructure/platform bottle-necks.

\item Application layer \hfill 

The application layer is the core service itself (the audio and video production core service) that must be adapted in order to offer full compatibility with previous mentioned layers and to become a cloud service.

\end{itemize}

All the items above are related to this project execution period that is aligned with specific goals of the i2CAT Foundation Audiovisual Unit \cite{i2catua}, which has specific research challenges such as to study and propose enhancements for networked media and interactive and immersive media. 

Concretely, this project is one of the next steps related to the LiveMediaStreamer framework project \cite{lmsGITHUB}, an open-source software developed in the Audiovisual Unit developer's team. One of this main functionalities is the capability of working as a software-based audio and video mixer, and this is the scenario that is going to be focused as the main service to be analysed.

Finally, to showcase that this document is organized as follows. Chapter \ref{A:stateOfTheArt} is about the state of the art of related technologies of the audiovisual content production, where the steps already done for convergence IP are briefly explained. Moreover, an introduction of the cloud concept and the topics related of this project are also shown. In chapter \ref{B:problemStatementAndProposal}, the problem statement are presented with a proposal solution, which are based on the topics already explained. Then, chapter \ref{D:application} (application), \ref{D:virtualization} (virtualization) and \ref{G:monitoringLayer} (monitoring) are each one focused on how the proposal solutions are developed. Then, in chapter \ref{H:platformDeploymentAndDemonstrations}
, specific deployments and demonstrations of the solutions proposed are carried out. And, last chapter \ref{C:conclusions} is about the conclusions obtained of the overall project.






\chapter{State of the art}\label{A:stateOfTheArt}

This chapter aims to provide a vision of how the audio-visual media production sector has been, and still is, carrying out IP convergence. 

Moreover, this chapter will be focused on the topics related to media production over an IP related environment and, specifically, the ones related on this project study and proposal.

\section{Audio-visual media content production}

As briefly as possible, this section aims to summarise the evolution of the IP convergence itself and how technologies are evolving to carry out such transformation at audio-visual media content production level.

Next subsections are also organized by enumerating potential technologies and standards following the OSI stack from the bottom layer.

\subsection{Physical and Data link layers}

Ethernet was standardised in 1983 and since then the standard has been increasing its speed rate from the initial 10 Mbps to 100 Gbps (foreseen 400 Gbps by IEEE P802.3bs Task Force), with currently easily affordable 10 Gbps and 40 Gbps interfaces. These rates seem to be enough to accommodate current broadcast formats (e.g.: HD at ~1.5 Gbps, 3G at ~3 Gbps and UHD at ~12 Gbps) and further innovations because of the nature of the packet technologies make them completely agnostic to the upper formats and indeed transparent for future formats in contrast with current media transport technologies which are completely bounded with the transported formats (i.e.: standard cable video formats used over broadcast environments). On another hand, Ethernet hasn't got any timing awareness or QoS assurance, so it makes difficult to accommodate current operation workflows over this technology. Nevertheless, because Ethernet is widely used in the IT industry, its use, as COTS switches, have motivated studies about the use of these switches in the broadcast industry deployments to validate specific necessary features as the latency deviation or packet loss. (REFERENCE: TEST LINK3)

At the same level, to address some of the inherent Ethernet limitations, Audio Video Bridging (AVB) appeared in 2011 which is a set of standard extensions to the Ethernet IEEE 802.1 focusing on timing and QoS guarantee within local area networks. Its approach is a plug-and-play platform to ease transition from current transport technologies to the newer ones using the same workflows, but the current version is still limited to local premises and limited topologies. Since November 2012, because more varied industry sectors joined the task group, a more general name, the Time Sensitive Networks (TSN), was created to carry on with the new developments.

On another hand, the emergence of the SDN paradigm (--REFERENCE--), separating the control and forwarding plane besides creating northbound interfaces to interact with external applications, enables new flexible and customised network operations and deployments. There are a lot of foreseen benefits from this approach but to be fully capable of support all type of streams some extensions should appear, such as a specific extension which has been released by the ONF to address timing restrictions known as OpenFlow Time Extension to OpenFlow 1.3.x ext340 (--REFERENCE--).

Furthermore, the Telecom Industry introduced the NFV concept (--REFERENCE--) in 2012 to enable the shifting from the hardware-centric approach to the software-centric one. In spite of not being thought for
broadcasting issues, this parading provides new possibilities to bring up new deployments and  enhance current workflows.

Related to above statements, the recent advances in chip designs by industry leaders (e.g.: Intel, Broadcom, Xilinx and Altera) have eased a strong movement towards consolidation of complex functions (e.g.: encoding, transcoding, conversion, \ldots ) into a single device, instead
several disparate platforms. Additionally, these hardware advances implied the chance of using software-centric frameworks, which provide for greater flexibility and customization from a business standpoint. Both advances are enabling a broader adoption of upcoming media technologies, at a reduced cost, without compromise on quality, flexibility and capability. Moreover, the disruptive nature of such advances can be seen in the mobile telephony market, such as the entry of open source mobile phone software (e.g.: Google’s Android OS) which, when coupled with low cost but powerful chipsets, has democratised the market allowing millions of people access to mobile phones and with it the Internet.

In parallel, regarding the specific timing and synchronising requirements of live media production, the standard IEEE 1588-2008 (--REFERENCE--), commonly known as PTPv2(--REFERENCE--), appeared covering several profiles to be used through several network environments. For instance, SMPTE(--REFERENCE--) published a draft profile SMPTE ST 2059 (1 and 2) (--REFERENCE--) defining a reference alignment to SMPTE epoch and there are studies analysing the application of PTP to broadcast environments under different circumstances (check this demo LINK4 --> Check also annex with ieee presentation!).

\subsection{Network and Transport layers}

On the network layer, IP is the de facto standard and within its protocol suite there are some solutions which help to transport media content efficiently. For instance, IP supports multicast paradigm operation using widely supported routing protocols (--REFERENCE--) (e.g.: DVMRP, PIM, IGMP, \ldots), but the computational and scalable complexity of these protocols tends to difficult and limit deployments.

In terms of QoS, IP has a mechanism known as ToS/DSCP (--REFERENCE--) which marks the header packets along their way to help mappings with lower-layer protocols (e.g.: Ethernet or MPLS) to implement QoS at the buffer level. Furthermore, IP networks have been evolving its own architectures from traditional hierarchic ones to flatter ones, such as the leaf-and-spine (--REFERENCE--), which is used in most of the nowadays big data centre deployments by facilitating horizontal data movement, which is useful for heavy load transactions between same level hosts.

On the transport layer, UDP has been preferred over TCP for real-time transport because of its connectionless and avoidance of unnecessary retransmission for live streams. As common known and basic extension, the RTP, which most deployed version is RFC 3550, was initially introduced to audio and video services using a timestamp field together with the protocol for control purposes (i.e.: RTCP). Recently, new extensions have appeared introducing new header options to support the adoption of services related to media production workflows. Concretely, RTP and RTCP have been proposed to accommodate media specific info over IP, answering to specific challenges.

\subsection{Session, Presentation and Application layers}\label{S:sessionPresentationApplicatio}

Encapsulation audio (AES67-2013) and video (SMPTE 2022-6) standards have appeared, around 2012 and 2013, to transport high-quality media signals over IP Networks.

In the audio field, a broad spectrum of proprietary solutions already exist (e.g.: Dante, RAVENNA and Livewire) (-- FOOTNOTE??REFERENCE--).

On the video side, SMPTE 2022-6 is focused on mapping SDI and HDSDI (opposite to raw video, audio and metadata mapping that is known as essence mapping --REFERENCE--) within IP packets.

Moreover, further specific solutions for manage packet loss recovery using FEC (SMPTE 2022-5) and a seamless protection system (SMPTE 2022-7) have appeared as potential robustness solutions.

In the middle of 2014, the Video Services Forum (VSF) has formed a new group (SVIP) looking at new encapsulation mechanisms for audio, video and ancillary data into IP without using SDI framing (raw data) to develop or recommend a standard for video over IP without SDI encapsulation.
They aim to study and document the requirements for video over IP/Ethernet within plant (i.e.: video, audio, ancillary data, bundles, timing, sequencing, identities and latency) in order to research over current and proposed solutions so that to report on gaps between requirements and existing solutions (especially regarding existing SMPTE 2022 Standards) and finally to propose scope for follow on activity, if required.

On another hand, the wide adoption of low-delay encoding (e.g.: JPEG2K, AVC, AVCi, VC-2) for high quality video stream could represent a new opportunity to reduce the bandwidth consumption in several scenarios. Likewise, high-compression mechanism as MPEG4 H264 or HEVC could be
useful to transport media content through very limited network resources scenarios (as Internet or cloud-based systems).

Moreover, specific efforts have risen to arrange specific challenges such as a networked media interface by Sony to carry a virtually lossless UHD/4K (12 Gbps non-compressed) over a single 10 GBE interface or specific implementations facing the switching-point issue. All of these are a useful starting point for future enhancements towards a global operational framework. (RESOURCES/REFERENCES??? LINK5)

Another important issue is the automation of the system to enhance flexibility for deployment set-up and maintenance. Here, some solutions as Zero-configuration networking (-- REFERENCE - LINK6 ) could contribute using well-known protocols as DHCP or DNS-SD to enable auto-configuration and streaming announcement, but to be implemented in an operational scenario a common approach should be defined.

In the media plane, protocols as RTSP for end-to-end session control or SDP for service description provide capabilities for the stream management. Furthermore, media wrappers aim to gather different types of programme media and associated information, as well as generically identify this information. Different media wrapper formats are in use at this time (e.g.: MXF --REFERENCE-->>>LINK7), but, for the media industry, it is important that the wrappers have characteristics like openness, extensibility and performance. The MXF (a SMPTE standard) is a “container” format, which supports a number of different streams of coded based by enabling interoperability between different platforms. This is by encoding in any type of video and audio compression formats, together with a metadata wrapper which describes the material contained within the MXF file. Also DDS, a machine-to-machines middleware standard from OMG (--REFERENCE--), could be used to enable interoperable media exchange between actors. At the same time, EBU has launched the FIMS (Framework for Interoperable Media Services --REFERENCE--) which intends to answer to different interoperability issues between SOA proprietary systems by defining an open, consensual framework with standardised interfaces.

Regarding the measurement of media transport over IP, VSF published in 2006 a document (RVOIM LINK8 ) to define the recommended metrics for Video over IP transport. The aim of the document is help in the monitoring, troubleshooting and equipment performance compliance to standards and specifications, verifying and measuring the delivered services statistics and equipment analysis and debug.

As could be inferred from the above statements, there are no current common approach to solve the whole challenge yet. To face this issue some initiatives have appeared lately. Many outstanding research initiatives are on the way but an interesting one was carried out by the BBC (--REFERENCE - LINK9) which tried to provide an operational framework for a live studio within their environment.

Likewise, in 2013 SMPTE, VSF and EBU created the JT-NM task force (JTNM) to drive the broadcasting industry towards a full IP adoption by providing guidelines to enable a successful migration. Currently, the JT-NM is working to develop a reference architecture to help all involved layers to agree on all cross issues whereas defining specific requirements over concrete use cases to uncover missing definitions to address the general scenario (--REFERENCE -- LINK10).

\section{Migration to cloud}

This section is a continuation of the previous one but focusing on how OTT content topics are giving new chances to enhance audio-visual media production to IP convergence, concretely within the cloud computing concept. 

!ADD GARTNERS HYPE CYCLE FIGURE AND COMMENTS ABOUT ITS STATE AND TENDENCE!

\subsection{Cloud computing}

Cloud computing describes the delivery of shared computing resources (software and/or data) on demand through the Internet and its benefits can be foreseen as it is defined by the NIST recommendations (--SEE ANNEX X OR REFERENCE--). So for reasons of flexibility, security, data protection, agility and cost, many organisations are migrating to cloud computing environments. 

Nowadays, cloud computing is defined by three fundamental models that are organized through application/service, platform and infrastructure layers.

\begin{figure}[htb]
\begin{center}
\includegraphics[width=0.35\textwidth]{./images/Cloud_computing_layers.png}
\caption{Cloud computing layers}
\label{F:cloudComputingLayers}
\end{center}
\end{figure}

Moreover, there are different deployment models depending on the product behind (e.g.: specific service or application), which are the resources from the entity that is offering or using such product. Main deployment models might be:

\begin{itemize}
\item Public: when applications/services run over a network that is open for public use, which may be free. The fact of being public/opened implies much more complexity in terms of security issues.
\item Private: when infrastructure is operated solely for a single organization, whether managed internally or by a third-party, and hosted either internally or externally. This cloud type might be similar in terms of architecture design from the public one.
\item Hybrid: when a composition of two or more clouds (private or public ones) are treated as distinct entities but are bound together, offering the benefits of multiple deployment models. Hybrid cloud allows to extend the capabilities of a cloud service by aggregation, integration or customization with another cloud service.
\end{itemize}

To point out that high-performance computing (e.g.: GPU based clouds --REFERENCE--) and software-defined networking (SDN) could shape where cloud is evolving and it could improve solutions to current cloud issues such as security (-- REFERENCE - National Security Agency and PRISM scandal), processing performance, full processing chain control through specific SLAs among others.

In many terms, the cloud concept is a key solution to help media producers create better content more quickly and there are lots of examples to focus on, but lets introduce the ones that tends to flexible and scalable ways to access the benefits that cloud computing brings to media production:

\begin{itemize}
\item Low-cost initial expenditures \hfill 

Media production tends to require an enormous initial investment in technology infrastructure and the technical staff to manage it. In that sense, cloud computing technology is that the creative industries are alleviated of the need to invest heavily in technology that would rapidly become obsolete. Cloud computing allows the media production industry to provision only the technology they need, when they need it, avoiding excessive CAPEX.

\item Cost forecasting\hfill 

Infrastructure as a Service (Iaas) prices are predictable and granularly treated. It allows prediction on a per project basis with detailed cost analysis precision. As done by many IaaS providers (e.g.: Amazon and Woowza), each resource used in a media production workflow is metered, and companies pay only for what they use.

\item Dynamic infrastructure deployment \hfill 

Cloud computing helps production entities take advantage by the on demand basis applied deployments. Media production companies can quickly provision servers to meet the demands of specific projects and shut them down when they are no longer needed.
\end{itemize}

Moreover, cloud computing can improve media production at many different media services requirements planes such as:

\begin{itemize}
\item Media asset management
\item Granular costs measurement
\item Cloud transcoding
\item High-speed file transfer
\item Automated content verification
\item Elastic deployment
\item Real-time and full monitoring
\item Video quality control
\end{itemize}

And, expected overall outcomes might be:

\begin{itemize}
\item Increased performance
\item Lowered costs
\item Improved cross collaboration
\end{itemize}

As a subchapter corollary, figure \ref{F:cloudComputingLayers} describes generic cloud computing layers, but it also refers to this project's layers in order. These are virtualization [\ref{SOA:Virtualization}], monitoring [\ref{SOA:monitoring}] and the service that is going to be deployed and tested which is implemented using the LiveMediaStreamer framework [\ref{SOA:LMS}].

\subsection{Virtualization}\label{SOA:Virtualization}

Virtualization, under computing environments, means creating a virtual version of any possible piece of actual hardware or software so that we can use system resources effectively. Besides hardware and desktop virtualization, which are the most known commercial concepts, it can be explained and organized in two different concepts:

(TO EXPLAIN EACH SUBITEM???)
\begin{itemize}
\item Types
\begin{itemize}
\item Data virtualization
\item Memory virtualization
\item Network virtualization
\item Storage virtualization
\item Security virtualization
\end{itemize}
\item Levels
\begin{itemize}
\item Application virtualization
\item Environment virtualization
\item Operating System (OS) virtualization
\item Networking virtualization
\end{itemize}
\end{itemize}

Cloud computing is usually strongly related and implemented with different kinds of virtualization. Many virtualization methods are commonly implemented at datacenters where platforms and services are going to be deployed over different infrastructure architectures. Nevertheless, deploying virtualization at data centers doesn’t automatically mean running over a cloud and it’s possible to deploy clouds without virtualization.

As well as cloud computing concept started to be widely used from 2000's, virtualization  technologies can be traced back to the 1960’s such as virtual desktops, and others can only be traced back a few years, such as virtualized applications (--REFERENCE-- LINK11).

Intro per portar a parlar dels containers i kvm -> es busca màxima flexibilitat, control, rapidesa d'actuació, etc. Parlar de PIRÀMIDE típica de cloud+virtualització i com altres mètodes poden modificar aquest "stack"

INTERESTS AND COMPARISONS (TENDENCES) -> TALK ABOUT KVM AND LXC
Containers represent one of the leading trends in computing today. With companies such as Docker, CoreOS, ClusterHQ joining industry giants like IBM, Red Hat, MIcrosoft and others in the rush to speed up the pace of container adoption. A recent study by DevOps.com and ClusterHQ showed that over 90\% of organizations have either looked at or plan to look at containers in the near future.



FINALLY:
Virtualization can increase IT agility, flexibility, and scalability while creating significant cost savings. Workloads get deployed faster, performance and availability increases and operations become automated, resulting in IT that's simpler to manage and less costly to own and operate.
Reduce capital and operating costs.
Deliver high application availability.
Minimize or eliminate downtime.
Increase IT productivity, efficiency, agility and responsiveness.
Speed and simplify application and resource provisioning.
Support business continuity and disaster recovery.
Enable centralized management.
Build a true Software-Defined Data Center.


\subsection{Monitoring}\label{SOA:monitoring}

Strongly related to cloud reliability is the monitoring concept. In order to reach maximum cloud reliability it's important to observe and check the progress and/or quality of key parameters over certain periods of time and to keep them under systematic review in order to create proper reactions, if required.

Therefore, this implies monitoring the cloud infrastructure (e.g.: servers, virtual or physical) and related services (e.g.: applications). Here appears the QoS (Quality of Service) and QoE (Quality of Experience) terms, respectively.

QoS is the monitoring and network-centric of underlying infrastructure components such as servers, routers and its network traffic. QoS metrics are generally device (e.g.: CPU and memory load, CPU temperature, disk space or HDD health) or transport-oriented (e.g.: packet loss, delay, bandwidth usage or jitter). 

Although QoS can be fully affordable due to the robustness and redundancy of current infrastructures (e.g.: back-up services, network rerouting and error correction), this doesn't mean that any end user might be feeling comfortable by using deployed services (e.g.: searching on a e-commerce webpage) over a high QoS infrastructure. Then, QoE monitoring evaluate the quality delivered to a user and it's done by analysing parameters when connecting to the service like a user. Therefore, QoE performance indicators are user-centric (e.g.: webpages response time and measuring video and audio quality (MOS)).

Common network monitoring protocols for distributed infrastructures management are:

(TO DEVELOP EACH ITEM)
\begin{itemize}
\item SNMP:
\item WMI: WBEM and CMI standards compatible implementation
\item NetFlow:
\end{itemize}  

Usually, these protocols are used to measure QoS, but there are complex algorithms that processes those QoS measurements parameters of interest in order to measure the QoE too. Nevertheless, there are specific applications to define and perform specific QoE measurements. This are known as bots or robots (EXAMPLE APPLICATION FOR SPECIFIC QOE MEASURES).

Going to what this project is about, the QoE measurements are relevant for audiovisual content services because bad network performance may highly affect the user's experience, mainly because these contents are compressed and coded, and have low entropy. Therefore, when designing systems for referenced analysis several elements in the video production and delivery chain may introduce distortion by degrading the content (i.e.: from the encoding system, transport network, access network, home network to end device).

There is also the referenceless analysis that is based on the idea that end users don't know about the original content. In this case, instead of measuring the QoE by comparing the original data to the delivered one, this is done by trying to detect artefacts (i.e.: blockiness, blur or jerkiness for video frames).

Typically, the evaluation of the QoE for audiovisual content provides users with a range of potential choices (i.e.: low, medium and high quality levels) that are currently widely accepted.

Obviously, the automation of critical cloud performance monitoring tasks is crucial for ensuring availability, providing efficient services and reducing common errors, costs and complexity. So, the use of OTT applications that processes such bank of data flows and displays outcome parameters of interest are crucial. PODEN SER DISTRIBUITS O CENTRALITZATS...

There are many applications and services that offer monitoring capabilities to be integrated to any kind new applications and infrastructures:

(FER TAULA SI SÓN APP O SERVEI I SI SON OBERTS O TANCATS)
\begin{itemize}
\item mrtg (tobias oeticker - rrdb) - prtg
\item cactix
\item opsview
\item monitis
\item new relic
\item zabbix
\item collectd + graphite
\item riemann 
\end{itemize}

So, thanks to monitoring evaluation service providers and network operators have the capability to minimize the storage and network resources by allocating only the resources that are required.

\subsection{Media}

transcoding, codecs, codification (nowadays) --> said at \ref{S:sessionPresentationApplicatio}

\section{LiveMediaStreamer (LMS) framework}\label{SOA:LMS}








\chapter{Problem statement and proposal}\label{B:problemStatementAndProposal}

As this title's chapter sample, the aim of this section is to provide a structured vision of the specific problems of each issue to be developed during this project.

Main issues are about to prepare and/or adapt current technologies to be ready for a cloud deployment. This fact implies studying existing technologies and tools in order to create specific microservices that will be interconnected between them. And, finally to carry out required developments in order to assure the goals behind and ease demonstrate the results.

\section{Architecture study}

Therefore, above all it's required to define the platform's architecture to prototype in order to have a global and generic insight. So, taking into account the pieces required for this project (i.e.: service layer, monitoring layer, virtualization/deploy layer), such architecture should contain the different high level layers as shown next:
\begin{figure}[htb]
\begin{center}
\includegraphics[width=0.5\textwidth]{./images/generalArch.png}
\caption{Generic platform architecture}
\label{F:genericPlatArch}
\end{center}
\end{figure}

The "LiveMediaStreamer API REST" layer will contain the service that will be offered to different and external applications in order to manage the "LiveMediaStreamer instance" layer by creating different audio and video production scenarios. Both layers are the core layers of the platform.

Moreover, in order to offer a centralized monitoring system, the "Statistics collector and display" layer becomes as the generic box for this requirement.

Finally, it is also required to provide an orchestrator that manages the deployment and distribution of the possible configurations for the previous introduced layers. This will be done thanks to the "Virtualization manager" layer.

\subsection{Virtualization}

This subsection aims to introduce the possibilities that different virtualization technologies could offer for this project requirements and to decide between one of them.

First of all, next points are showing which are the expected outcomes for using virtualization and what should fulfil the selected technology:

\begin{itemize}
\item To manage and maintain a system of small pieces of services (assure a microservice architecture based pattern)
\item To have flexibility in order to quickly load (e.g.: start, restart, stop) required instances (e.g.: to assure real-time scalability) and to deploy any possible required scenario/configuration.
\item To offer ease to continuous develop and deploy the different parts of the architecture
\item To have version like system for having different version tags for the architecture modules (e.g.: a development and a production box of the same API REST service)
\item To assure full compatibility for the core layers' operating system (right now only Linux environments are supported)
\item To assure full compatibility for the hardware to work with (mainly x86 processors are in the scope)
\end{itemize}

Therefore, it is also required to use an as much lightweight as possible technology.

Under Linux environments there are many virtualization options (most of them proprietary) to analyse, but lets focus on the ones that are open-source and have wider and active communities behind.

These are KVM with QEMU and LXC with Docker:

\begin{itemize}
\item KVM (Kernel-based Virtual Machines) \hfill

It's a FreeBSD and Linux kernel module that offers a full virtualization solution for Linux on x86 hardware containing virtualization extensions (Intel VT or AMD-V). It consists of a loadable kernel module that provides the core virtualization infrastructure and a processor specific module.
Usually, KVM runs with the QEMU (Quick Emulator) which is a complete and standalone emulating suite that performs hardware virtualization.

KVM with QEMU is able to offer virtualization for x86, PowerPC, and S/390 guests. For instance, when the target architecture is the same as the host architecture, QEMU can make use of KVM particular features in order to do not emulate CPU nor memory.

\begin{figure}[!htb]
\begin{center}
\includegraphics[width=0.9\textwidth]{./images/KVM.png}
\caption{QEMU with KVM or hypervisor type2 to type1}
\label{F:KVMandQEMU}
\end{center}
\end{figure}

Figure \ref{F:KVMandQEMU} showcases how KVM can convert a type2 hypervisor (i.e.: QEMU) into a type1 hypervisor (known as a bare metal hypervisor) which increases overall application performances. 

\item LXC (Linux Containers) \hfill

It's an operating-system-level virtualization environment for running multiple isolated Linux systems (known as containers) on a single Linux central host.

Linux kernel itself provides the cgroups functionality that allows limitation and prioritization of resources (CPU, memory, block I/O, network, etc.) without the need for starting any virtual machines, and namespace isolation functionality that allows complete isolation of an applications' view of the operating environment, including process trees, networking, user IDs and mounted file systems.

Nowadays virtualization tendencies are focusing on the Docker project, which is a platform that provides an additional layer of abstraction and automation of operating-system-level virtualization on Linux, Mac OS and Windows.
\begin{figure}[!htb]
\begin{center}
\includegraphics[width=0.9\textwidth]{./images/LXC.png}
\caption{Docker with LXC}
\label{F:DockerAndLXC}
\end{center}
\end{figure}

LXC with Docker mean resource isolation features of the Linux kernel to allow independent containers to run within a single Linux instance, avoiding the overhead of starting and maintaining virtual machines. Moreover, Docker project automates the deployment of applications inside software containers and offers different tools to manage them (i.e.: CLI, API and file configurations), as shown in figure \ref{F:DockerAndLXC} 
\end{itemize}

Thus, it seems that the technology that best suits this project requirements is the Docker project. Main reasons are the capabilities of ease maintain, test and quickly deploy each container. Moreover, ---APENDIX X--- demonstrates that the performance offered in front of the Virtual Machine (i.e.: QEMU + KVM) one is rather better.

Finally, it's important to point out that another keypoint of the Docker solution is that lets intercommunicate containers that are in the same physical server without exposing network layer. This means much more security and performance over other virtualization solutions (at least of the simplicity point of view when intercommunicating instances).

\subsection{Monitoring layer}

Monitoring layer section aims to provide an insight of what are the minimum requirements for what monitoring is required in this project and which are the available tools of interest.

Following the same criteria as this project tends to, the tools that are going to be focused on are the ones that have an open-source philosophy behind. So, most of the tools and services around the Internet are directly ruled out.

As done in the virtualization problem statement section, it is required to showcase main requirements for the monitoring layer implementation:

\begin{itemize}
\item To ease its networkability.
\item To offer full control for specific configuration requirements and to not depend on external/enterprise services. To be fully deployable.
\item To be as lightweight as possible in terms of:
\begin{itemize}
\item Processing capacity
\item Memory utilization
\item Network throughput
\end{itemize}
\item To fit into a microservice architecture pattern.
\item To be supported under a Linux environment.
\item To offer much more capabilities than the required for future requirements (this is a requirement because this platform development is supposed to be continued after this project)
\item To support RRD (Round Robin Database) as a DBMS (Database Management System) in order to assure data storage size from the beginning.
\end{itemize}

Next list to showcase takes into account the data that should be gathered:

\begin{itemize}
\item CPU and memory usage per process which is part of the platform.
\item Network usage (per process involved and per media)
\item Internal core service performance parameters (i.e.: LiveMediaStreamer core performance):
\begin{itemize}
\item Processing time
\item Losses ratio
\end{itemize}
\end{itemize}

Moreover, in order to be fitted into a microservices pattern, it's required to split the monitoring layer. The proposal, as usually done in many monitoring systems is as follows:
\begin{itemize}
\item Gathering layer \hfill

It only will be responsible of the gathering, distribution and storing of the metrics.

\item Display layer

It only will be responsible of the data display as a function of user-specific queries.

\end{itemize}
To point out that a possible alert layer is treated as a future work because it becomes out of scope. The main reason is that this project is focused on measuring and demonstrating that such platform is feasible, next steps would be to compose a set of alarms and actuators system in order to become as automated as possible.

Once previous steps have been shown, it's time to analyse technologies of interest in order to decide between one of them.

So, the following list shows some of the analysed tools that might fit project's requirement as previously exposed:

\begin{itemize}
\item Munin \hfill

Cross-platform web-based network monitoring and graphing tool designed as a front-end application. Written in Perl.

\item RRDTool \hfill

Known as an industry standard, high performance data logging and graphing system for time series data. Written in C and runs under GNU/Linux, Windows and AIX platforms.

\item Collectd \hfill

Small daemon which collects system information periodically and provides mechanisms to store and monitor the values in a variety of ways. Written in C and support any unix-like O.S.

\item Graphite \hfill

Tool for monitoring and graphing the performance of computer systems, which collects, stores, and displays time series data in real time.

\end{itemize}

Many other solutions have been discarded due to depend on enterprises roadmap or due to not fit under the exposed requirements. (REFERENCE Comparison of network monitoring systems WIKIPEDIA) 

Finally, Collectd and Graphite are the selected tools. Main reasons are due to being fully configurable and its core design and philosophy. Collectd has a data distribution system based on a push model and it can be single deployed (gathering and display layer), but it is going to be used with the storage and display tool Graphite. This fact is due to assure high performance for the containers where the Collectd will be gathering and transmitting metrics through UDP, and because of both tools lead to create a microservice architecture monitoring layer. Moreover, Graphite software will be deployed as a centralized storing and displaying tool which will be feed from many Collectds. Another reason to select Collectd is due to being fully compatible with Docker. 

Finally, Graphite has also been selected because it offers a HTTP API which ease its scalability. 

So, the monitoring layer is proposed to be designed as shown next:

\begin{figure}[htb]
\begin{center}
\includegraphics[width=0.9\textwidth]{./images/mlap.png}
\caption{Monitoring layer architecture proposal}
\label{F:MLAP}
\end{center}
\end{figure}

\subsection{Application layer}


\section{Architecture proposal}
\section{Task planning}




\chapter{Application}\label{D:application}

The main goal of this chapter is to develop the tasks related to the application, including to prepare LMS to be deployed as a cloud service and to give support for internal and external monitoring.

\section{REST API}

As mentioned previous chapters, it is required to develop an API ready to be used over cloud environments in order to ease creating specific and new applications over the LMS framework, and thus to demonstrate the viability of this thesis prototype.

Nowadays, the common and widely used format for cloud services intercommunication is JSON, as it is also used for the TCP socket API of the LMS framework. Therefore, this API middleware is going to follow such requirement.

A suitable technology to work with JSON formatted messages is Node.js \cite{nodejs} which is widely known for its good performance. Node.js is an open source, cross-platform runtime environment for server-side and networking applications. It provides an event-driven architecture and a non-blocking I/O API that optimizes application's throughput and scalability. This technology is commonly used for real-time web applications. 

Working with Node.js means avoiding serialization of the JSON messages by increasing services intercommunication performance (i.e.: less computational cost and less processing time). 

A common Noje.js framework for developing web applications and REST APIs is Express.js \cite{expressjs}. It is the de-facto standard server framework for Node.js. So, our middleware development is going to use Express.js routing system (URIs with HTTP request methods like GET, POST, PUT and DELETE).

Figure \ref{F:restAPI} illustrates the software structure proposal for developing the HTTP RESTfull API middleware, which is responsible for translating the TCP socket API of the LMS framework.

\begin{figure}[!htb]
\begin{center}
\includegraphics[width=1\textwidth]{./images/RESTAPI.png}
\caption{RESTfull API middleware architecture}
\label{F:restAPI}
\end{center}
\end{figure}

\begin{itemize}
\item HTTP REST API layer \hfill

This layer handles HTTP queries from external applications. It implements specific routes to handle specific HTTP queries. The first implementation will not implement multiple LMS management but single, as shown in Figure \ref{F:restAPI}.

\item Interface layer \hfill

This layer handles the body messages from the previous layer's HTTP queries and manipulates them in order to create an as much generic as possible API by adapting the messages to be sent through following TCP socket layer.

\item TCP socket layer \hfill

This layer is the responsible for sending and receiving JSON-formatted TCP socket messages to and from the targeted LMS instance.
\end{itemize}

As presented in Section \ref{SOA:LMS} and detailed in Appendix \ref{ANX:lmsarchfull}, there are two different management layers: the generic one and the filter specific one. So, by following this organization, the proposed API's structure is as described in Appendix \ref{ANX:RESTAPI}.

Note that this API is not implementing persistence\footnote{Persistence, in computer science, refers to the characteristic of state that outlives the process that created it. This is usually solved by storing the state's application as data structures in databases.} because the state (managed through 'State' method) is given by the LMS instance itself. The unique sign of persistence is regarding the LMS host and port which the middleware is connected to (managed through 'Connect' and 'Disconnect' methods). Higher levels of persistence should be implemented by external applications, which implies specific scenarios and requirements (i.e.: specific persistence).

For more details about how this is structured and the overall middleware is implemented check Appendix \ref{ANX:sourceCodes} with the code. 

\section{Network metrics}

Network metrics could be treated as external metrics. This is because these metrics are specially dependent from sources which transmit to LMS, the receivers from LMS and the state of the network itself. Obviously, the performance of the LMS affects to the metrics gathered too, but this effect is intended to be minimized, at least in a gathering and presentation of the metrics' point of view.

\subsection{Input network metrics}

As mentioned in Section \ref{B:appLayerCH2}, input network metrics are going to be implemented by carrying out methods re-implementations of methods provided by the Live555 library, which is the library that will manage network streams.  

\begin{figure}[!htb]
\begin{center}
\includegraphics[width=0.45\textwidth]{./images/SourceManager.png}
\caption{Input network metrics structure}
\label{F:inms}
\end{center}
\end{figure}

By following the LiveMediaStreamer architecture structure, input network implementations are going to be developed inside the 'liveMediaInput' structure. Specifically, a new class is developed, called 'SCSSubsessionStats'. This class is managed by the 'StreamCleanState' class, which is a class related to each stream 'Session' class managed by the 'SourceManager' class. This last class is a 'HeadFilter' class. Figure \ref{F:inms} shows the inter-class structure.

By initializing new RTP or RTSPClient sessions (i.e.: network inputs), a group of subessions is associated per each stream (i.e.: an RTP session has one subsession associated and the RTSPClient session has as many subsessions as accepted from the SDP that defines different RTP sessions).

When a new subsession is set, a new RTPReceiverStats class is automatically initialized. This Live555 object implements RTCP stats measurement which are only required to be treated outside. This is done at SCSSubsessionState, which creates a new schedule to periodically measure and save current state (a default granularity of 1 second is set). The implemented method is called periodically as shown in Appendix \ref{ANX:appALG} Section \ref{inmm}. This implementation prepares the metrics that are going to be presented when a new state query is received. The code in Appendix \ref{ANX:appALG} Section \ref{inmm} shows how metrics from the Live555 library are obtained. For a more detailed insight of the overall implementation see Appendix \ref{ANX:sourceCodes}. The metrics that are presented per each new state query are shown next (a default granularity of 1 second is set):

\begin{itemize}
\item Bitrate: maximum, minimum and average in kbps.
\item Packet loss percentage: maximum, minimum and average.
\item Inter-packet gap: maximum, minimum and average in milliseconds.
\item Jitter: maximum, minimum and current inter-packet gap variation in microseconds.
\end{itemize}

All these metrics are measured through previous algorithm shown which is executed each second (it is the default value set for all metrics gathered indeed). Specifically, bitrate is measured by dividing the total number bytes, received during each scheduled period, by the elapsed time given by the Live555 library implementation:

\begin{equation}\label{E:bitrate}
average\ bitrate (kbps) = \frac{kbits\ received}{elapsed\ seconds}
\end{equation}

The maximum and the minimum bitrates are the last maximum and minimum average bitrates obtained, and this is done for all other metrics.

Jitter is measured as the estimate of the statistical variance of the RTP data interarrival time inserted in the interarrival jitter field of reception reports (in microseconds), and this is already internally done by the Live555 library. So, what is presented is the current jitter value at the beginning of each new schedule.

Another metric that might be of interest is the delay from the stream source to the LMS instance but it is discarded due to not being offered from Live555 library. Moreover, it has been discared to be implemented at SCSSubsessionStats class level due to is computational cost and complexity to develop such requirement. Note that this metric is not relevant because network performance problems can be detected through other metrics already gathered (i.e.: jitter and packet loss ratio).


\subsection{Output network metrics}

As done in previous section, output network metrics are going to be implemented by carrying out re-implementations of methods given by the Live555 library.

This implementation has been much more difficult due to not having control of the creation or deletion of the RTPSink class of the Live555 library. Previous developments before the final version were based over the RTPSink re-implementation already done per each OnDemandServerMediaSubsession (Live555 library class), which is also re-implemented by QueueServerMediaSubsession (LMS framework class). But the implementation was still losing the specific RTPSink instance of specific subsession.

In order to continue with the development, an e-mail was sent to the Live555 developers mailing list and a solution was provided by Ross Finlayson as shown in Appendix \ref{ANX:emailRoss}.

The best option, as suggested by Ross, was to re-implement the RTCPInstance class per each inheriting class of the OnDemandServerMediaSubsession class, specifically the inheriting classes of the QueueServerMediaSubsession class. 

\begin{figure}[!htb]
\begin{center}
\includegraphics[width=0.45\textwidth]{./images/SinkManager.png}
\caption{Output network metrics structure}
\label{F:onms}
\end{center}
\end{figure}

Figure \ref{F:onms} shows the relationship between the SinkManager class with both possible types of connections (RTP or RTSP). Each Connetion object has a map of objects of the re-implemented RTCPInstance class, called ConnRTCPInstance (i.e.: RTCP instances per each connection). Then, each time a new specific RTP connection or any QueueServerMediaSubsession (RTSP connection, from RTSP server) is created, a ConnRTCPInstance is associated in order to start gathering the statistics offered from Live555 library. This is shown in Appendix \ref{ANX:appALG} Section \ref{onmm}, with the code of the method that is periodically called (default periodicity value is set to 1 second). In this case, the delay metric is gathered from the Live555 library.

The metrics are:

\begin{itemize}
\item Bitrate: maximum, minimum and average in kbps.
\item Packet loss percentage (ratio): maximum, minimum and current.
\item Round trip delay: maximum, minimum and current in milliseconds.
\item Jitter: maximum, minimum and current inter-packet gap variation in microseconds.
\end{itemize}

All these metrics are measured by following the same expressions used for the input network metrics.

\section{Pipeline metrics}

The following metrics are presented as the internal metrics. Please, note that the measurement of these metrics interfere the overall performance of the LMS framework. Therefore, to achieve the minimum possible computational cost is a must. But, first of all, let us pick up the example figure of a pipeline from Appendix \ref{ANX:lmsarchfull} in order to showcase the internal pipeline structure of the LMS framework. As described in Appendix \ref{ANX:lmsarchfull}, this will help understanding the following two implementations.

\begin{figure}[!htb]
\begin{center}
\includegraphics[width=1\textwidth]{./images/LMSpipelineBasicOne.png}
\caption{LMS framwork's internal pipeline structure example}
\label{F:lmsps}
\end{center}
\end{figure}

Note that in Figure \ref{F:lmsps} the arrows are the queues which interconnect each filter's writer with another filter's reader. Writers and readers are subclasses of the IOInterface class.

\subsection{Delay}

This metric is related to the time that data (i.e.: a video frame, an audio sample,\ldots) takes to be processed from an origin point to an end point by an unique given path, and this is measured from a given and required time: the data's timestamps. Data inside the LMS source code is known as a "frame", which can be an audio frame (i.e.: sample), a H264 video NAL unit, a raw frame, \ldots 

The delay time is not measured per each filter in order to not affect the overall performance, but each measured time involves an origin filter which resets the timestamps. These origins are the Head (i.e.: receiver) and OneToMany (i.e.: audio and video mixer) filters. These filters reset the frames' timestamps in order to reach and control an internal synchronization, and this is due to the fact that these filters have many outputs (i.e.: writers) or many inputs (i.e.: readers) and they require synchronizing their outputs in order to assure one point of time control inside the LMS framework, whatever the scenario configured.

So, in order to measure the delay it is important to note that a pipeline is not composed by an unique path (see Appendix \ref{ANX:lmsarchfull} for clarifications) but multiple paths. This fact implies that it is not possible to measure an unique overall delay time per frame which goes over the pipeline\footnote{There is a special case when there is only a path that defines the pipeline itself (e.g.: a path for only transcoding video with one quality: a receiver, a decoder, a re-sampler, an encoder and a transmitter, all connected within a path that defines the pipeline of this specific scenario)}, or at least it is not suitable for performance issues, but it can be done for external applications which know the scenario configuration and gathers such metrics. Therefore, this is solved by splitting the measurements by paths. And this is an optimal measurement: the delay is given by the differential time measured by the last reader of the path (i.e.: the reader of the destination filter). Appendix \ref{ANX:appALG} Section \ref{pdmm} shows the method which implements the measurement inside the Reader class.

In summary, the code measure the average delay of a frame by a given window time (default is configured to 1 second) with a resolution of microseconds. And this means measuring from the origin time, which, as said, is set by the origin filters (i.e.: the beginning of a path, starting from a writer), to each reader of the path (i.e.: each filter). But, the delay time presented is from the beginning (i.e.: initial writer of the path) until the last reader. Figure \ref{F:lmsps} illustrates different path examples.

\subsection{Losses}

This metric follows a similar criteria as the previous one. This is solved by measuring at the same point, but it is done when flushing frames at the reader side (as presented in Section \ref{B:appLayerCH2}). In order to reach minimal computational cost, the measurement is just a counter of the overall data losses when calling the flush methods. What is done is a method encapsulation by defining a parent method that is just incrementing its reader counter and then it calls the specific flush implementation per each data/frame type.

In order to properly present it to external applications it is required to be presented at a path level as done for the delay metrics. So, in this case it is only required to sum up the overall losses of the path's readers.

It is important to note that this metric is not referenced to a total data processed or at any time point, but measuring its continuity over time permits detect losses with different thresholds. This means that if this value is incrementing gradually then the system is not working properly. Such detections might imply fast increments on its continuity (differential increase). 

Note that Pipeline metrics are measured when flushing frames (discarding) or when a reader is able to remove a frame from its belonging queue. 

Finally, once previous developments are carried out a first step to offer a cloud real-time media production service is achieved. This means, offering a RESTful API and a status layer for monitoring external and internal performance of the platform.
\chapter{Virtualization}\label{D:virtualization}


dir que desktop virtualization requereix de molts recursos per poder encapsular una app... no es requereix que sigui a nivell gràfic... explicar prèviamanet les caracteristiques del software i l'arquitectura que hauria de seguir.
\chapter{Monitoring layer}\label{G:monitoringLayer}

As already discussed, in order to properly manage a cloud computing environment it is strongly required to use monitoring tools in order to gather information of interest by improving the environment itself or by finding out issues and solving them as fast as possible.

This chapter aims to showcase how the selected monitoring tools can fit in the architecture type that has been proposed and how they can be used. This means, preparing the environment to support Collectd (i.e.: monitoring and gathering) and Graphite (i.e.: storing and presenting) tools.

Moreover, to point out that such tools are helpful to demonstrate that LiveMediaStreamer framework could be deployed as the core of a real-time media production platform (see chapter \ref{H:platformDeploymentAndDemonstrations}).

So, to remark that the fact of creating small and reusable containers is the main goal of this chapter and, of course, the goal of this platform architecture to prototype. And, thanks to the selected monitoring tools, which are lightweight and ease configuration flexibility, this issue might be properly solved.

Then, a proposal of the monitoring architecture in a more detailed description (regarding figure \ref{F:MLAP}) is shown in figure \ref{F:maex}.

\begin{figure}[htb]
\begin{center}
\includegraphics[width=0.9\textwidth]{./images/monitArchProp.png}
\caption{Detailed monitoring architecture}
\label{F:maex}
\end{center}
\end{figure}

Figure \ref{F:maex} showcases the relationship between different containers and the whole Collectd+Graphite deployment. Following sections are explaining it:

\section{Monitoring containers}

This section is based on how Collectd can be configured and deployed in order to monitor and properly gather metrics of interest.

\subsection{From container point of view}

From a container point of view and by following the premise to build containers as reusable as possible what is proposed to implement is a container that has the goal to gather the logged stats from an LMS container. This, as shown in figure \ref{F:maex}, implies sharing a Docker volume (as introduced in previous chapter \ref{D:virtualization}) from LMS container to the Collectd client which is using the tail plugin as an input. Moreover, in order to send specific logged metrics to the Collectd server container it is also using the network plugin as done in previous Collectd client configuration.

But, first of all, it is also required to be built in a container. This specific Docker file is shown in APPENDIX \ref{ANX:dockerFiles} section \ref{ANX:dockerFiles5}.

Then, what is done in this case is to specify a bash script to run as CMD. This runs collectd but after envtpl python's package sets the environment parameters:
\begin{verbatim}
#!/bin/bash
envtpl /etc/collectd/collectd.conf.tpl
collectd -C /etc/collectd/collectd.conf -f
\end{verbatim}

Thanks to the envtpl package it is possible to run the container with specific environment variables in order to configure following parameters:

\begin{itemize}
\item LMS NAME \hfill

this will be used to identify the LMS instance in a container which Collectd is monitoring.
\item GRAPHITE HOST \hfill

this is to set the address of the remote/local container where the Collectd server is listening and pushing the metrics inside the Graphite's tools.
\item GRAPHITE PORT \hfill

this is the port where the Collect server and Graphite's tools container is listening to.
\end{itemize}

And these parameters are set in the Collectd configuration file as shown in APPENDIX \ref{ANX:dockerFiles} section \ref{ANX:dockerFiles5} (among the specific plugins).

This Collectd configuration example file is loading specific system loggers plugins as inputs to be sent through the network plugin to the Collectd server.

But, in next chapter \ref{H:platformDeploymentAndDemonstrations} is shown an example of use of the Collectd tail plugin by using regular expressions. And this, as shown in figure \ref{F:maex}, this tail plugin is listening in an specific folder which is shared through the Docker's volume functionality with the LMS container, which logs its metrics in the same volume.

Finally, the Docker run command where the specific environment variables are set should be as shown next:
\begin{verbatim}
$ docker run -it -e "LMS_NAME=lms" \
	-e "GRAPHITE_HOST=<IP address>" -e "GRAPHITE_PORT=25826" \
	--rm --name cdc \
	-p 25826:25826/udp gerardcl/lms-collectd-client
\end{verbatim}

This will start immediately sending the defined container stats to the Graphite container specified by the environment parameters. Following section showcases the Graphite side to be deployed.

So, this example of deployed container for isolated Collectd clients is a key point in the general monitoring architecture due to the fact of being easyly configurable and reusable.

\subsection{From OS point of view}

The fact of using Collectd means a wide community behind, which probably have already developed required functionalities. And this is the case: in order to monitor each of the containers that a host OS might have it can be solved by configuring already existing plugins or similar tools for Collectd from Docker community. 

The selected tool is using the stats API introduced since Docker 1.5 version. And, concretely, the reported container's stats are:
 
\begin{itemize}
\item Network
\item Memory usage
\item CPU usage
\end{itemize}

The plugin is called "collectd-docker" and its documentation can be found in GitHub \cite{cdcontainer}. But, in order to follow the idea of clustering Collectd it has been modified in order to use the network plugin instead the write graphite plugin. So, the fact of being running on the same OS it requires changing the UDP ports through the metrics are sent by avoiding port binding issues. Moreover, previous Collectd client is also configured as a proxy server by re-configuring the network plugin as shown next: 

\begin{verbatim}
<Plugin network>
  Server "{{ GRAPHITE_HOST }}" "{{ GRAPHITE_PORT | default("25826") }}"
  Listen "*" "25827"
  Forward true
  ReportStats true
</Plugin>
\end{verbatim}

Therefore, it is not a plugin itself but a Docker container, which listens the docker daemon socket of the system (as proposed in figure \ref{F:maex}) and monitors each one of the containers running.

This container implements the same envtpl python package in order to define specific environment parameters such as the collectd server host and port (check its documentation for a detailed configuration explanation). So, in order to showcase how this and the previous Collect client are interconnected the following Docker run commands are shown:

\begin{itemize}
\item Collectd tail client (from the container point of view):
\begin{verbatim}
docker run -it --rm --name cc -v /home/gerardcl/logs/:/home/lms/logs \
-p 25826:25826/udp -e "LMS_NAME=lmsAVMixingStats" \
-e "GRAPHITE_HOST=192.168.1.140" \
gerardcl/lms-collectd-client
\end{verbatim}
\item Collectd docker socket API reader (from the OS point of view):
\begin{verbatim}
docker run -v /var/run/docker.sock:/var/run/docker.sock \
-e GRAPHITE_HOST=127.0.0.1 -e COLLECTD_HOST=lmsOS \
-e COLLECTD_DOCKER_APP=lmsAVMixStats -e GRAPHITE_PORT=25827 \
-it --rm --name collector --net="container:cc"  \
gerardcl/lms-collectd-collector
\end{verbatim}
\end{itemize}

So, here, what is done in order to avoid port binding issues is to setup the Docker collectd collector container to use the same network environment of the Collectd tail client.

\section{Showcasing monitoring}

This section is focused on the storage and presentation side of the metrics already logged and gathered.

So, as already introduced, this is proposed to be done within a container built with a Collectd server (network data inputs served to Graphite) and a Graphite system (storing and presenting stats).

As commonly said by the Collectd and Graphite community, installing and configuring Graphite isn't at all so easy than installing and configuring Collectd. This fact is shown in the Docker file of the \ref{ANX:dockerFiles} section \ref{ANX:dockerFiles6} in order to build such container, as shown in figure \ref{F:maex}.

In this case, specific "collectd" and "graphite" users are created (among other environment configurations as shown). Then, a bunch of different and specific configuration files for graphite are added (i.e.: ADD command) to their specific configuration folders. And, finally, specific command executions for database (i.e.: sqlite3 \cite{sqlite}, which is required for specific features for the Graphite web application) synchronization and other final configurations required for Graphite are also done. Regarding Collectd, it is installed in a similiar way as previously done but it is now configured as shown in \ref{ANX:dockerFiles} section \ref{ANX:collectdFiles2}.

So, here, Collectd is configured as a server by listening from anywhere at the default port for Collectd clustering. Moreover, the "write graphite" plugin is loaded in order to work as an output to the Graphite's Carbon tool, which receives the data to be stored in the Whisper RRD of the Graphite installation.

In this case, it is required to configure this container to be able to run multiple processes (i.e.: Graphite and Collectd). Therefore, by following previous section of how to run multiple processes inside a container in chapter \ref{D:virtualization}, the Supervisord system is used. This time it's configured by splitting the processes configurations into two parts, the Collectd and the Graphite. This last is configuring the Graphite web application and the Graphite's Carbon cache.

\begin{itemize}
\item Collectd: \hfill

\begin{verbatim}
[program:collectd]
user=collectd
directory=/
command=collectd -C /etc/collectd/collectd.conf -f
stdout_logfile=/var/log/supervisor/%(program_name)s.log
stderr_logfile=/var/log/supervisor/%(program_name)s_error.log
\end{verbatim}
\item Graphite web: \hfill

\begin{verbatim}
[program:graphite-web]
user=graphite
directory=/opt/graphite/webapp/
command=/opt/graphite/env/bin/gunicorn -w 1 -b 0.0.0.0:8080 \
	--pythonpath /opt/graphite/webapp/graphite graphite_wsgi
stdout_logfile=/var/log/supervisor/%(program_name)s.log
stderr_logfile=/var/log/supervisor/%(program_name)s_error.log
\end{verbatim}

\item Carbon cache: \hfill

\begin{verbatim}
[program:carbon-cache]
user=graphite
directory=/
env=PYTHONPATH=/opt/graphite/lib/
command=/opt/graphite/bin/carbon-cache.py --debug start
stdout_logfile=/var/log/supervisor/%(program_name)s.log
stderr_logfile=/var/log/supervisor/%(program_name)s_error.log
\end{verbatim}
\end{itemize}

An important configuration that must be decided and configured by knowing the requirements for why monitoring is required implies defining the storage schemas, which detail retention rates for storing metrics by following what was introduced in chapter \ref{B:problemStatementAndProposal}, the Round-Robin Database storage type. So, in order to work over a real-time media production platform it is important to achieve as much time accuracy as possible regarding the specific metrics of interest (e.g.: bandwidth usage, losses, pipeline delays, \ldots). There are more files regarding the Graphite's tools configuration too, but most of them are set to default values which are the recommended ones. 

Before going to present results and comment the outcomes (chapter \ref{H:platformDeploymentAndDemonstrations}), it is important to showcase example commands for both Collectd client and Collectd server + Graphite's containers:

\begin{verbatim}
$ docker run -it -e "LMS_NAME=lms" -e "GRAPHITE_HOST=<IP of >" \
--rm --name collectd -p 25826:25826/udp gerardcl/lms-collectd-client
\end{verbatim}

\begin{verbatim}
$ docker run -it --rm --name graphite -p 25826:25826/udp -p 8080:8080 \
gerardcl/lms-collectd-graphite
\end{verbatim}

As shown, the collectd (i.e.: Collectd client) container is sending to the default port (i.e.: UDP protocol) of the graphite host's (Collectd server and Graphite tools) container. Moreover, the graphite container opens HTTP port 8080 in order to enable browsers to get access to its web application anywhere. 

\chapter{Platform deployment and demonstrations}\label{H:platformDeploymentAndDemonstrations}

In order to demonstrate that the LiveMediaStreamer is a suitable tool to be used as the core framework of a cloud real-time media production platform it's required to showcase how it performs over the cloud. Therefore, it's important to demonstrate how it performs over the proposed architecture. And, previous chapters have implicitly helped to deploy and demonstrate it as shown in this chapter.

\section{Platform deployment}

In order to demonstrate how LMS suits to become a proper tool it's proposed to deploy two scenarios with different complex degrees.

\begin{itemize}
\item Isolated deployment \hfill

The main goal of this deployment is to demonstrate how LMS performs inside a Docker container by comparing its performance in the same O.S. but without running inside a container (i.e.: system installation).

In this scenario deployment LMS is configured to act as a transcoder service. This means applying one pipeline per stream type (i.e.: one video and one audio paths).

\item Generic scenario deployment \hfill

This scenario deployment aims to showcase a suitable and a as much generic as possible cloud real-time media production scenario. Therefore, it is proposed to configure LMS to receive eight streams (i.e.: four audio and four video streams), mix them and transmit them through RTP/RTSP. 
\end{itemize}

The environment where the deployments are done has the following characteristics:

\begin{table}[htb]
\caption{Deployment environment characteristics}
\begin{center}
\begin{tabular}{|c|c|}
\hline
{\bf Parameter} & {\bf Value} \\ \hline \hline
Hardware type        & Sony Vayo laptop \\ \hline
CPU        & Intel core i7-3632QM at 2.20 GHz  \\ \hline
RAM        & 6 GB (4 + 2) DDR3 \\ \hline
Operating system        & XUbuntu 14.04 - 64 bits (x86 64)  \\ \hline
Kernel version        & 3.13.0-55-generic  \\ \hline
Docker version        & 1.6.8  \\ \hline
\end{tabular}
\label{T:dec}
\end{center}
\end{table}

As seen, the deployment hasn't been carried out in any type of specific server or high performance cluster environment. This is mainly due to demonstrate flexibility on the deployment (i.e.: in a laptop) and portability of the platform (not only the cloud itself). And, all of these characteristics are achieved thanks to the performance of the platform itself and, concretely, the LMS (the core).

Finally, to point out that all tests have been carried out in a local area network with a router with both laptops connected (see figure\ref{F:idsc} and figure \ref{F:gdsc}) at one gigabit of bandwidth. Moreover, the measurements have been carried out during 10 minutes and a second of granularity.

\subsection{Isolated demonstrations}

In order to demonstrate results of interest what is done is to implement a C/C++ script which configures the LMS framework as shown in figure \ref{F:idsc}. Moreover, in order to test how it performs the pipeline metrics are logged once per second (i.e.: pipeline losses and delay) and gathered by a Collectd client container properly configured. Then the Collectd client sends the data to the Graphite container. 

\begin{figure}[!htb]
\begin{center}
\includegraphics[width=0.95\textwidth]{./images/isolatedScenario.png}
\caption{Isolated demonstration's scenario configuration}
\label{F:idsc}
\end{center}
\end{figure}

The Collectd client container, which reads from the volume where the LMS container is logging its metrics, is using the \texttt{tail} plugin (previously explained in chapter \ref{G:monitoringLayer}) with specific regular expressions in order to parse the metrics from the LMS logs (See APPENDIX XXX for this specific collectd configuration)

So, both isolated scenarios are the same but one is running the LMS on the system and the other is running the same configured LMS but containerized.

The second O.S. running is the one which runs the Docker container with the Collectd+Graphite tools. Moreover, this O.S. acts as the receiver of the transcoded streams through RTSP protocol and also as the transmitter of the source stream.

Therefore, let's group and show the results of interest which are mainly focused on the pipeline performance metrics (i.e.: internal LMS performance):

\begin{figure}[!htb]
  \begin{center}
    \begin{subfigmatrix}{2}
      \subfigure[System installation]
         {\includegraphics{./images/testStats/testStatsOS/8cpuIdleAVG.png}\label{SF:S1}} 
      \subfigure[Containerized]
         {\includegraphics{./images/testStats/testStatsDocker/8cpuIdleAVG.png}\label{SF:S2}} 
    \end{subfigmatrix}
    \caption{Isolated scenarios - CPU usage}
    \label{F:isoCPU}
  \end{center}
\end{figure}

Figure \ref{F:isoCPU} showcases the CPU usage gathered at the Collectd client side and presented for the Graphite web GUI for both isolated scenarios. Regarding system installation the CPU average usage of the averages given by Graphite is around 4,015\% and around 4,111\% for the containerized one. So, there is not so difference about running system installation or the same application containerized. 

In order to continue demonstrating that the fact of deploying a service platform within a containerized environment let's continue showcasing results of interest:

\begin{figure}[!htb]
  \begin{center}
    \begin{subfigmatrix}{2}
      \subfigure[System installation - Video path]
         {\includegraphics{./images/testStats/testStatsOS/vAVGdelayMS.png}\label{SF:S3}} 
      \subfigure[Containerized - Video path]
         {\includegraphics{./images/testStats/testStatsDocker/vAVGdelayMS.png}\label{SF:S4}} 
      \subfigure[System installation - Audio path]
         {\includegraphics{./images/testStats/testStatsOS/aAVGdelayMS.png}\label{SF:S3}} 
      \subfigure[Containerized - Audio path]
         {\includegraphics{./images/testStats/testStatsDocker/aAVGdelayMS.png}\label{SF:S4}}
    \end{subfigmatrix}
    \caption{Isolated scenarios - average pipeline processing time}
    \label{F:isoappt}
  \end{center}
\end{figure}

Figure \ref{F:isoappt} showcases the average pipelines delay introduced for the LMS system which in both video and audio cases is almost the same. Regarding the video, system installation reaches an average of 216,8 milliseconds and the containerized reaches an average of 215,9 milliseconds. Then, regarding audio system installation reaches an average of 25,6 milliseconds and the containerized reaches an average of 25,2 milliseconds.

\begin{figure}[!htb]
  \begin{center}
    \begin{subfigmatrix}{2}
      \subfigure[System installation - Video path]
         {\includegraphics{./images/testStats/testStatsOS/vLostBlocs.png}\label{SF:S3}} 
      \subfigure[Containerized - Video path]
         {\includegraphics{./images/testStats/testStatsDocker/vLostBlocs.png}\label{SF:S4}} 
      \subfigure[System installation - Audio path]
         {\includegraphics{./images/testStats/testStatsOS/aLostBlocs.png}\label{SF:S3}} 
      \subfigure[Containerized - Audio path]
         {\includegraphics{./images/testStats/testStatsDocker/aLostBlocs.png}\label{SF:S4}}
    \end{subfigmatrix}
    \caption{Isolated scenarios - pipeline accumulated lost blocs}
    \label{F:isoaplb}
  \end{center}
\end{figure} 

Regarding pipeline losses, as shown in figure \ref{F:isoaplb} both pipelines within both system installation and containerized scenarios don't introduce any data loss.

And this is a sign that LMS is a good option to work with not only on system installation but in a containerized environment in order to be a portable service over a cloud infrastructure.

\subsection{Generic scenario demonstration}

This last scenario aims to be a generic and basic example demonstration of audio and video production in a cloud environment. Figure \ref{F:gdsc} showcases how the scenario is configured.

\begin{figure}[htb]
\begin{center}
\includegraphics[width=0.95\textwidth]{./images/genericScenario.png}
\caption{Generic demonstration's scenario configuration}
\label{F:gdsc}
\end{center}
\end{figure}

This demonstration is quite similar than the other but this time LMS is only configured and running inside a container. This LMS configuration is also a C/C++ script which congiures the framework, as shown in figure \ref{F:gdsc}, inside the LMS container, specifically.

So, in this case what is deployed is an audio and video mixer which receives four audio streams and four video streams encoded with OPUS and H264 codecs, respectively, which are streamed through its standard RTP encapsulation (i.e.: specific payload headers).

Regarding the audio mixing, all input streams are mixed with logarithmic mixing algorithm in order to not saturate the signal of the resulting audio stream. Then, regarding the video mixing, the HD inputs (i.e.: 1280x720 pixels of resolution) are mixed as shown next:

\begin{figure}[!htb]
\begin{center}
\includegraphics[width=0.90\textwidth]{./images/outAVmix.png}
\caption{Generic scenario - video mixing configuration result}
\label{F:outVMix}
\end{center}
\end{figure}

All video inputs are resized (through the pre-resampler filter to the video mixer, see \ref{F:gdsc}) to the half of its size in order to fill into a resulting HD video stream as shown in figure \ref{F:gdsc}.

So, once the scenario is introduced, let's focus on the obtained results of interest as done in previous section. This is focusing on the pipeline performance parameters.

The CPU usage of the containerized audio and video mixer obtained by averaging the averages shown in figure \ref{F:gsavgcpu} is about the 23,02\%.

\begin{figure}[!htb]
\begin{center}
\includegraphics[width=0.90\textwidth]{./images/testAVMix/AVMixCPU.png}
\caption{Generic scenario - average CPU usage}
\label{F:gsavgcpu}
\end{center}
\end{figure}

Then,the audio and video average pipeline delay introduced for this scenario's configuration is as shown in figure \ref{F:gsavgpt}. Concretelly there are two path groups, the "recevier to mixer" paths and the "mixer to transmitter" path (see figure \ref{F:gdsc}).

\begin{figure}[!htb]
  \begin{center}
    \begin{subfigmatrix}{2}
      \subfigure[Audio paths]
         {\includegraphics{./images/testAVMix/AVMixAudioAVGdelay.png}\label{SF:S5}} 
      \subfigure[Video paths]
         {\includegraphics{./images/testAVMix/AVMixVideoAVGdelay.png}\label{SF:S6}} 
    \end{subfigmatrix}
    \caption{Generic scenario - paths average processing time}
    \label{F:gsavgpt}
  \end{center}
\end{figure}

So, the average delay introduced for the audio "receiver to mixer" paths averages is about 8,6 milliseconds and the audio "mixer to transmitter" path average delay is about 23.1 milliseconds. Then, by adding the maximum average path (16,2), a total average value of 39,3 milliseconds of average processing time regarding the audio pipeline is achieved. Then, regarding the same calculus but about the video pipeline path's, an average of 9,89 milliseconds is obtained by averaging the "receiver to mixer" video paths average processing times. And, by adding the average of the "mixer to transmitter" path processing time of 67 milliseconds to the maximum average obtained in the "receiver to mixer" path (12,8 ms) a total average of 79,8 milliseconds of delay introduced for the video pipeline is obtained.

\begin{figure}[!htb]
  \begin{center}
    \begin{subfigmatrix}{2}
      \subfigure[Audio paths]
         {\includegraphics{./images/testAVMix/AVMixAudioLostBlocs.png}\label{SF:S7}} 
      \subfigure[Video paths]
         {\includegraphics{./images/testAVMix/AVMixVideoLostBlocs.png}\label{SF:S8}} 
    \end{subfigmatrix}
    \caption{Generic scenario - paths accumulated lost blocs}
    \label{F:gsalb}
  \end{center}
\end{figure}

In figure \ref{F:gsalb} the audio paths accumulated lost blocs remains to zero meaning that the audio pipeline is not discarding any data at any filter, which is an important fact because losing any byte of audio means hearing it. 

Then regarding the video "receiver to mixer" paths there aren't accumulated data losses. But, "mixer to transmitter" path reaches around 52.308 lost data blocs. The fact of having lost data blocs is due to the transitory period of the mixer filter. However, this accumulated losses remains constant, meaning that there are no more data blocs lost.

So, although this scenario is being deployed in a i7 processor laptop, it's able to real-time mix four couples of audio and video streams without issues.

Finally, to point out that the appearing burst in some figures are due to the fact of transmitting origin streams in a pseudo-live mode, which means audio and video loops. Therefore, these bursts appear when restarting the origin streams.
\cleardoublepage
\phantomsection
\chapter*{Conclusions}\label{C:conclusions}

Escriure aquí les conclusions del projecte. 

LINKS CONCLUSIONS

http://windowsitpro.com/blog/going-beyond-virtualization-private-cloud

vGPU

virtualization and monitoring are the key start points to develop a ...

monitorització - next steps about different monitoring profiles (debug, active, passive,...)

comentar les capes desenvolupades i en quina capa del cloud computing layers aniria i d'aqui a tractar una a una
\cleardoublepage
\phantomsection
\chapter*{Acronyms}

\begin{table*}[htb]
\centering
\begin{tabular}{p{0.2\textwidth} p{0.7\textwidth}}
\hline
AVB & Audio Video Bridging \\
\hline
AVC & Advanced Video Coding \\
\hline
AVCi & Advanced Video Coding - Intra \\
\hline
COTS & Commercial Off-The-Shelf \\
\hline
DVMRP & Distance Vector Multicast Routing Protocol \\
\hline
GPU & Graphics Processing Unit \\
\hline
IEEE & Institute of Electrical and Electronics Engineers \\
\hline
IGMP & Internet Group Management Protocol \\
\hline
JPEG2K &  Joint Photographic Experts Group in 2000\\
\hline
IP & Internet Protocol \\
\hline
MPLS & Multiprotocol Label Switching \\
\hline
NFV & Network Functions Virtualization \\
\hline
ONF & Open Networking Foundation \\
\hline
OTT & Over The Top \\
\hline
PIM & Protocol Independent Multicast \\
\hline
PTPv2 & Precision Time Protocol version 2 \\
\hline
QoS & Quality of Service \\
\hline
SDN & Software-Defined Networking \\
\hline
SLA & Service Level Agreement \\
\hline
SMPTE & Society of Motion Picture and Television Engineers \\
\hline
SVIP & Define and research requirements for Video over IP without SDI encapsulation \\
\hline
RFC & Request for Comments \\
\hline
RTP & Real-time Transport Protocol \\
\hline
RTCP & Real-time Control Protocol  \\
\hline
TCP & Transmission Control Protocol \\
\hline
ToS/DSCP & Type Of Service / Differentiated Services Code Point \\
\hline
TSN & Time Sensitive Networks \\
\hline
UDP & User Datagram Protocol \\
\hline
VC-2 & Dirac Pro - Video Codec Level 2 \\
\hline
VSF & Video Service Forum \\ 
\hline
\end{tabular}
\end{table*}




%\input{chapters/organitzacio}



%%%  BIBLIOGRAFIA
%%%%%%%%%%%%%%%%%%%%%%%%%%%%%%%%%%%%%%%%%%%%%%%%%%%%%%%%%%%%%%%%%%%%%%%%%%

%%% Per la bibliografia hi ha 2 opcions: generarla amb la utilitat BibTeX 
%%%                                      o fer-la ''a ma''
%%% NOTA: podeu trobar facilment informació sobre BibTeX a:
%%%  http://www.ctan.org/tex-archive/biblio/bibtex/contrib/doc/

%%% OPCIO 1: BibTeX (recomanat) -> descomentar les comandes seguents:
%\bibliographystyle{unsrt}   %% Estil de bibliografia EETAC
%\cleardoublepage
%\phantomsection
% Indicar aqui el(s) fitxer(s) que contenen la bibliografia
%\bibliography{fitxer1,...,fitxerN}  
%\pdfbookmark{Bibliografia}{sec:biblio}

%%% OPCIO 2: bibliografia manual
%%%
%%% L'argument d'entrada es el numero de referencies que s'inclouen
\cleardoublepage
\phantomsection
\begin{thebibliography}{2}

%% Llibres:  Autor/s (cognoms i inicials dels noms), títol del llibre (en cursiva), editor, ciutat i any de publicació. Quan es cita el capítol d'un llibre s'ha d'indicar el títol del capítol (entre cometes), el títol del llibre (en cursiva) i els números de pàgines amb la primera i la darrera incloses.

%%  Exemple de capitol en llibre
\bibitem{prova1} 
Cognoms-autor, Inicial-nom.
``Títol del capítol''. {\it Títol del llibre}.
(Editor. Ciutat. Any publicació): pagina1--paginaN.

%%  Exemple de d'article en revista
\bibitem{prova2} 
Cognoms-autor, Inicial-nom.
``Títol de l'article''. {\it Títol de la revista}.
{\bf volum}(numero),
pagina1--paginaN. (Any publicació) 

\bibitem{LiveMediaStreamer framework GitHub}
LiveMediaStreamer framework GitHub. Homesite: https://github.com/linux-wpan/linux-wpan-next.git

\end{thebibliography}

%%%%%%%%%%%%%%%%%%%%%%%%%%%%%%%%%%%%%%%%%%%%%%%%%%%%%%%%%%%%%%%%%%%%%%%%%%
%%%%%%                           APENDIXS                         %%%%%%%%
%%%%%%%%%%%%%%%%%%%%%%%%%%%%%%%%%%%%%%%%%%%%%%%%%%%%%%%%%%%%%%%%%%%%%%%%%%
\pagestyle{empty}  % no tocar

%% Descomentar una de les dues línies següents, en funció de:
%%  a) els apendixs s'encuadernaran apart (amb portada) 
%%  b) els apendixs s'enquadernen amb el mateix projecte (sense portada). 
%% Recordeu que si tot el document (amb apèndixs) excedeix les 100 pagines 
%% s'ha d'enquadernar a part
%\appendix\ambportada
\appendix\senseportada


%%%%%%%%%%%%%%%%%%%%%%%%%%%%%%%%%%%%%%%%%%%%%%%%%%%%%%%%%%%%%%%%%%%%%%%%%%
%%%%%% INCLOURE A PARTIR D'AQUI TOTS ELS CAPÍTOLS DELS APENDIXS   %%%%%%%%
%%%%%%%%%%%%%%%%%%%%%%%%%%%%%%%%%%%%%%%%%%%%%%%%%%%%%%%%%%%%%%%%%%%%%%%%%%

\chapter{LiveMediaStreamer architecture}\label{ANX:lmsarchfull}

links to official LMS web site which I've developed during August 2015
\chapter{Source codes}\label{ANX:sourceCodes}

Full source codes can be found in the GitHub's web page of the Media Internet Area's developers team of the i2CAT Foundation:

\begin{itemize}
\item \href{https://github.com/ua-i2cat/LMStoREST}{HTTP REST API} \hfill

This is the source code repository for the RESTfull API middleware.\hfill

\url{https://github.com/ua-i2cat/LMStoREST}

\item \href{https://github.com/ua-i2cat/liveMediaStreamer}{LiveMediaStreame framework} \hfill

This is the source code repository for the LiveMediaStreamer framework. \hfill

\url{https://github.com/ua-i2cat/liveMediaStreamer}

Concretely:
\begin{itemize}
\item \href{https://github.com/ua-i2cat/liveMediaStreamer/tree/development/src/modules/liveMediaInput}{Statistics for network inputs} \hfill

\url{https://github.com/ua-i2cat/liveMediaStreamer/tree/development/src/modules/liveMediaInput}

\item \href{https://github.com/ua-i2cat/liveMediaStreamer/tree/development/src/modules/liveMediaOutput}{Statistics for network outputs} \hfill

\url{https://github.com/ua-i2cat/liveMediaStreamer/tree/development/src/modules/liveMediaOutput}
\item \href{https://github.com/ua-i2cat/liveMediaStreamer/blob/development/src/IOInterface.cpp}{Statistics for pipeline metrics} \hfill

\url{https://github.com/ua-i2cat/liveMediaStreamer/blob/development/src/IOInterface.cpp}
\end{itemize}


\end{itemize}


\chapter{Exchanged e-mails with Live555 developer mailing list}\label{ANX:emailRoss}

This appendix is showing the e-mail conversation I had with the CEO and CTO of the Live555 library, which helped a lot for developing a proper solution to gather the Live555 statistics (i.e.: network statistics).

E-mail sent:

\begin{quote}
\begin{verbatim}
Sender: Gerard Castillo Lasheras <gerard.castillo@i2cat.net>
Receiver: LIVE555 Streaming Media - development & 
									use <live-devel@ns.live555.com> 
Hi Ross,

I'm implementing statistics on our software (liveMediaStreamer framework) 
and I'd like to have access to the RTPTransmissionStatsDB. But, I do not
see how to get the RTPSink object (which has the RTPTransmissionStatsDB 
and its stats).

Which should be the proper way to get the RTPSink object related to my
OnDemandServerMediaSubsession childs? I've seen that 
OnDemandServerMediaSubsession has a friend classe StreamState 
which has the RTPSink associated but, anyway, 
I'm not able to have access to it.

Thanks in advance,

Kind regards,
\end{verbatim}
\end{quote} 

The reply:
\begin{quote}
\begin{verbatim}
Sender: Ross Finlayson <finlayson@live555.com>
Receiver: LIVE555 Streaming Media - development & 
									use <live-devel@ns.live555.com> 
									
First of all, note that while a "OnDemandServerMediaSubsession" 
object refers to a track of streamable media, a "RTPSink" object 
refers to a receiving client (or possibly multiple clients if 
"reuseFirstSource" is True).  So there’s (in general) a 
one-to-many relationship between "OnDemandServerMediaSubsession" 
and "RTPSink".  Thus, it doesn’t make sense to talk about 
*the* RTPSink object for your "OnDemandServerMediaSubsession".

However…
There are at least two possible ways to get access to the 
"RTPSink" objects:

1/ Note the pure virtual function "createNewRTPSink()" that you 
have implemented in your "OnDemandServerMediaSubsession" subclass.  
You can use your implementation of this function to get access 
to the "RTPTransmissionStatsDB" for the new "RTPSink", after 
you’ve created it.
The drawback of this approach, though, is that you don’t know when 
the "RTPSink" object later gets deleted, so - if you’re not 
careful - you may end up holding a reference or pointer to a 
"RTPTransmissionStatsDB" that has been deleted.

2/ Define a subclass "myRTCPInstance" of the "RTCPInstance" 
class.  Then, in your "OnDemandServerMediaSubsession" subclass, 
reimplement the "createRTCP()" virtual function to create a 
"myRTCPInstance" object, rather than a "RTCPInstance" 
object.  Note that "createRTCP()" contains a "sink" 
parameter, pointing to a "RTPSink", from which you can 
get the "RTPTransmissionStatsDB".
The advantage of this approach over approach 1/ is that 
- by defining a subclass of "RTCPInstance", you can learn when 
the "RTPInstance" object gets deleted, and thus when the 
"RTPSink" object gets deleted.  (The "RTCPInstance" object 
always gets deleted immediately before the "RTPSink" 
object.)  Thus, you can use your "myRTCPInstance" destructor 
to figure out when the "RTPTransmissionStatsDB" should no 
longer be used.
									
\end{verbatim}
\end{quote} 

\chapter{Docker cheat sheet}\label{ANX:csd}

This is a continually expanded GitHub repository where Docker users contribute with specific usages of the Docker technology:

\href{https://github.com/wsargent/docker-cheat-sheet}{Docker cheat sheet}

Current version of previous Docker cheat sheet is attached within next pages:

\includepdf[pages=-]{./appendix/wsargent-docker-cheat-sheet-blob-master-README.pdf}


%%%%%%%%%%%%%%%%%%%%%%%%%%%%%%%%%%%%%%%%%%%%%%%%%%%%%%%%%%%%%%%%%%%%%%%%%%
%%%%%%%%%%%%%%%%%%%%%%%%%%%%%%%%%%%%%%%%%%%%%%%%%%%%%%%%%%%%%%%%%%%%%%%%%%
%%%%%%%%%%%%%%%%%%%%%%%%%%%%%%%%%%%%%%%%%%%%%%%%%%%%%%%%%%%%%%%%%%%%%%%%%%
% i  aixo es tot! ;)
\end{document}





